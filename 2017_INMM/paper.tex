%\documentclass{proc}  % 2-column format
\documentclass[12pt]{paper}
%\documentclass{ntmanuscript}
%\documentclass[review]{elsarticle}
\usepackage{mathptmx} % Nearly Times New Roman
\usepackage[acronym,toc]{glossaries}
\newacronym{MIT}{MIT}{the Massachusettes Institute of Technology}
\newacronym{UW}{UW}{University of Wisconsin}
\newacronym{UM}{UM}{University of Michigan}
\newacronym{Sandia}{Sandia}{Sandia National Laboratories}
\newacronym{US}{US}{United States}
\newacronym{HEU}{HEU}{highly enriched uranium}
\newacronym{GWe}{GWe}{gigawatt electrical}
\newacronym{LEU}{LEU}{low enriched uranium}
\newacronym{U}{U}{uranium}
\newacronym{SWU}{SWU}{separative work unit}
\newacronym{CNERG}{CNERG}{Computational Nuclear Engineering Research Group}
\newacronym{DRE}{DRE}{dynamic resource exchange}
\newacronym{UOX}{UOX}{uranium oxide}
\newacronym{MOX}{MOX}{mixed oxide}
\newacronym{SNM}{SNM}{special nuclear material}
\newacronym{WGP}{WGP}{weapons-grade plutonium}
\newacronym{NPT}{NPT}{Nuclear Nonproliferation Treaty}
\newacronym{IAEA}{IAEA}{International Atomic Energy Agency}
\newacronym{CTBT}{CTBT}{Comprehensive Nuclear-Test-Ban Treaty}
\newacronym{START}{START}{Strategic Arms Reduction Treaties}
\newacronym{FMCT}{FMCT}{Fissile Material Cutoff Treaty}
\newacronym{NWPM}{NWPM}{Nuclear Weapons Pursuit Model}
\newacronym{NWS}{NWS}{Nuclear Weapons State}
\newacronym{NNWS}{NNWS}{Non-nuclear Weapons State}
\newacronym{CVT}{CVT}{Consortium for Verification Technology}
%\newacronym{<++>}{<++>}{<++>}
%\newacronym{<++>}{<++>}{<++>}
%\newacronym{<++>}{<++>}{<++>}
%\newacronym{<++>}{<++>}{<++>}

%\makeglossaries
%%%%%%%%%%%%%%%%%%%%%%%%%%%%%%%%%%%

\usepackage{color}
\usepackage{subcaption}
\usepackage{graphicx}
\usepackage{booktabs} % nice rules for tables
\usepackage{microtype} % if using PDF
\usepackage{xspace}
\usepackage{listings}
\usepackage{textcomp}
%\usepackage{ulem}

% Page length commands go here in the preamble
\setlength{\oddsidemargin}{-0.25in} % Left margin of 1 in + 0 in = 1 in
\setlength{\textwidth}{7in}   % Right margin of 8.5 in - 1 in - 6.5 in = 1 in
\setlength{\topmargin}{-.75in}  % Top margin of 2 in -0.75 in = 1 in
\setlength{\textheight}{9.2in}  % Lower margin of 11 in - 9 in - 1 in = 1 in



\definecolor{listinggray}{gray}{0.9}
\definecolor{lbcolor}{rgb}{0.9,0.9,0.9}
\definecolor{burgundy}{rgb}{0.5, 0.0, 0.13}
\definecolor{burntorange}{rgb}{0.8, 0.33, 0.0}
\definecolor{chromeyellow}{rgb}{1.0, 0.65, 0.0}
\definecolor{darkred}{rgb}{0.55, 0.0, 0.0}

\lstset{
    %backgroundcolor=\color{lbcolor},
    language={C++},
    tabsize=4,
    rulecolor=\color{black},
    upquote=true,
    aboveskip={1.5\baselineskip},
    belowskip={1.5\baselineskip},
    columns=fixed,
    extendedchars=true,
    breaklines=true,
    prebreak=\raisebox{0ex}[0ex][0ex]{\ensuremath{\hookleftarrow}},
    frame=single,
    showtabs=false,
    showspaces=false,
    showstringspaces=false,
    basicstyle=\scriptsize\ttfamily\color{green!40!black},
    keywordstyle=\color[rgb]{0,0,1.0},
    commentstyle=\color[rgb]{0.133,0.545,0.133},
    stringstyle=\color[rgb]{0.627,0.126,0.941},
    numberstyle=\color[rgb]{0,1,0},
    identifierstyle=\color{black},
    captionpos=t,
}

\newcommand{\code}[1]{\lstinline[basicstyle=\ttfamily\color{green!40!black}]|#1|}
\newcommand{\units}[1] {\:\text{#1}}%
\newcommand{\SN}{S$_N$}
\newcommand{\cyclus}{\textsc{Cyclus}\xspace}
\newcommand{\Cyclus}{\cyclus}
\newcommand{\citeme}{\textcolor{red}{CITE}\xspace}
\newcommand{\TODO}[1] {{\color{red}\textbf{TODO: #1}}}%

\newcommand{\comment}[1]{{\color{green}\textbf{#1}}}

%%%%%%%%%%%%%%%%%%%%%%%%%%%%%%%%%%%
\begin{document}


%\begin{frontmatter}
\title{State-Level Decision-Making In Cyclus to Assess Multilateral Enrichment}

% Authors. Separated by commas
\author{
  Meghan B. McGarry$^1$,
  Drew Buys$^1$,
  Paul P.H. Wilson$^1$}


\date{}
% Institutes of the authors
\institution{$^1$Department of Nuclear Engineering and Engineering Physics \\
University of Wisconsin - Madison}
\institution{$^2$Sandia National Laboratories}
% Information concerning the person submitting the manuscript
%\submitter{Meghan B. McGarry}
%\submitteraddress{1500 Engineering Drive, Madison, WI, USA}
%\submitteremail{mbmcgarry@wisc.edu}

% No more than three keywords, though each can be a phrase
%\keywords{fuel cycle, simulatiom, non-proliferation}
\maketitle

% I. Motivation
% x   -  Identify Factors that motivate
% x  II. Benchmark model against historical data:
% x A. Develop historical database
% x   - What sources?
% x   - What model to convert to 10pt scale?
% x        - Conflict
% x B. Determine relative weighting of factors
% x   - Calculate values for historical database
% x   - PCA to determine relative weights
% ** C. Table of State Score Results
% III. Develop forward model 
%   - From score to a likelihood, using historical data
% IV. Limitations of model
%   - small dataset
%   - threshold value for proliferation
%   - no good model for scientific network
% V. Future work
%   - Apply to case study (JCPOA?)
% VI Appendix of factor conversions

\begin{abstract}

 Proposed treaties and agreements that aim to reduce the spread of nuclear weapons are often stalled by skepticism regarding their efficacy. For example, the concept of multilateral enrichment, in which multiple states co-own and operate an enrichment facility, has the potential to reduce the spread of enrichment technology. However, detractors point to the improved international networking opportunities inherent in multinational organizations as a risk factor for increased proliferation. A framework to compare the relative risk between a multilateral enrichment paradigm and the status quo, on a regional scale, can help inform the discussion and potentially identify ways to reduce global risk of nuclear proliferation. As part of the Consortium for Verification Technology, the Cyclus fuel cycle simulator is being used as a test-bed for the development of such new technologies and approaches to treaty verification. Cyclus is a systems-level nuclear fuel cycle simulator that models the interactions between actors in the nuclear arena. While designed to track the flow of nuclear material between facilities, Cyclus also incorporates an innovative Facility-Institution-Region hierarchy that can capture the dynamics of state-level interactions. Drawing on social science literature to identify factors that motivate states to pursue weapons programs, we have developed a regional proliferation model that captures causes and effects of state-level nuclear weapons proliferation. The model identifies eight key factors that influence a state’s decision to pursue nuclear weapons. These factors include motivations internal to the state, such as military spending and governing structure, as well as interactive factors such as conflict between states. Historical data is used to identify the relative importance of these factors and translate them into a likelihood of pursuing a weapon. The model also provides a feedback mechanism such that acquisition of a nuclear weapon by one state influences the decision-making of the other states. This model will be used to assess the effectiveness of policy approaches, such as multilateralization of the fuel cycle, that seek to reduce the regional risk of proliferation over time.

\end{abstract}


%\end{frontmatter}


\section{Introduction}
\label{s_motive}

%% I have pivoted this paragraph to de-emphasize the tie between energy knowledge and bomb knowledge. Is it ok now or still to much implied?  If you don't like how it sounds feel free to edit. I am having trouble striking the right balance.
Nuclear expertise is rapidly expanding around the world as demand for energy increases steadily. Because nuclear energy is clean and carbon-neutral, climate change concerns further tilt the scales making nuclear power appealing to a growing number of countries \cite{mooney_why_2014}.  China is already investing heavily in nuclear power, planning to triple its generating capacity from 19 \gls{GWe} to 58 \gls{GWe} by 2020 \cite{_china_2014}.  As climate change becomes increasingly important with respect to national security, the perception of the risk inherent to nuclear energy is decreasing and states are embracing nuclear energy as a reliable large-scale source of carbon-neutral energy.  However, the expansion of nuclear power amplifies nuclear security concerns, because the same technologies used to produce nuclear fuel can also be exploited in the pursuit of nuclear weapons.  Moreover, in the 70 years since nuclear bombs were dropped on Hiroshima and Nagasaki, the knowledge and technology required to make these weapons has proliferated around the globe \cite{feiveson_unmaking_2014}. There are now nine states that have developed their own nuclear weapons either through indigenous research or transfer of knowledge from existing programs. As nuclear power becomes more ubiquitious, it becomes ever more important to meaningfully decouple the nuclear expertise required for the pursuit of energy from that of nuclear weapons.  

\subsection{Sensitive Parts of the Nuclear Fuel Cycle}

Two nuclear technologies are of of particular concern for proliferation, uranium enrichment and plutonium reprocessing.  Uranium enrichment is required for the once-through fuel cycles that are dominant around the world today, and used exclusively in the \gls{US}.  A once-through fuel cycle includes a source of natural uranium such as a mine, and is comprised primarily of non-fissile 99.3\% $^{238}U$, with only 0.7\% fissile $^{235}U$ that is able to undergo nuclear fission. Concentrations of 3-5\% fissile $^{235}U$ are typical for fueling a nuclear power reactor.  (Research reactors use higher levels of enrichment and there are ongoing efforts to phase out those that use enrichments above 20\%).  Enrichment facilities are used to increase the concentration of $^{235}U$ from natural stock to the desired amount.  Fuel is then burned in a nuclear reactor and the remaining material, which includes the majority of the original $^{238}U$, short- and long-lived fission products, and $\sim$1\% Pu (239 and 240), is then stored as waste.  The enrichment phase of the fuel cycle is a poliferation concern because in principle it can be used to increase the concentration of $^{235}U$ up to the 90\% or more typically used to make a nuclear weapon \cite{_military_2014}.
%% There is much debate about making weapons out of lower enrichments, probably not worth including here. -- %% I've heard a bit about using lower enrichments for bomb-making, but not much. Maybe we discuss in Ann Arbor as well.

Plutonium reprocessing is a proliferation concern because the technique can be used either to make recycled fuel or to make weapons-grade fissile material.  Several countries are developing nuclear reactors that can accomodate recycled fuel, providing the possibility of a closed fuel cycle in which the burning of nuclear fuel would at the same time generate new nuclear fuel \cite{_processing_2015}.  Recycled fuel is plutonium-based rather than uranium-based, and is made by separating the components of spent uranium fuel to extract the plutonium concentrations of fissile $^{239}Pu$. This material can then be blended with uranium to make \gls{MOX} fuel. (\gls{MOX} can also be made by sourcing the plutonium from decomissioned weapons). The concentration of $^{239}Pu$ depends on the amount of time the fuel was burned in the reactor, and can be upwards of 50\%.  Specially designed irradiation of uranium fuel can produce a plutonium component with $^{239}Pu$ concentrations up to 93\%, known as \gls{WGP}.  Reprocessing has been considered in the \gls{US} at several times over the past half-century.  However, a host of political, economic, environmental and strategic concerns have pushed the issue of reprocessing out of the technical realm and it has become a contentious political topic, currently the \gls{US} is pursuing only basic science research in this field \cite{rossin_policy_????, editorial_adieu_2009}.
%% Do you want to mention currently-under-construction US MOX facility for converting WGP to commercial MOX fuel? It may never be completed, though.
%% I think I'll leave it for question/answer session

\subsection{Use of Treaties to Curtail Proliferation}

While it has not proven possible to prevent the spread of nuclear knowledge entirely, international treaties have been used in an attempt to minimize it.  The \gls{NPT}, which has been signed by 190 states including the original five nuclear weapons states, has codified a set of rules and norms for allowing the peaceful pursuit of nuclear energy \cite{_treaty_????}.  The \gls{NPT} created the \gls{IAEA}, whose role is to verify compliance with the treaty by periodically inspecting facilities related to nuclear technology.  Other relevant treaties include \gls{CTBT}, which placed a moratorium on testing nuclear weapons, and the \gls{START} in which the \gls{US} and Russia agreed to nuclear arms reductions \cite{_treaty:_????, department_of_State_new_2010}. (The \gls{CTBT} has been signed by 164 states but has not yet entered into force).

These treaties have done much to prevent the spread of nuclear weapons knowledge, but they do not address the weapons production capabilities of states that already posess nuclear weapons.
%% These treaties are not designed to prevent the spread of nuclear knowledge, just nuclear weapons knowledge.
%% Sorry, I didn't mean to imply otherwise.
A potential \gls{FMCT} would place limits on the amount of weapons-grade fissile material that each signatory state could stockpile, possibly including current stockpiles in the case of weapons states.  However, a major unresolved issue is the difficulty of developing verification techniques to ensure compliance \cite{_fissile_2013}.  Furthermore, measuring nuclear material for treaty verification is itself a sensitive issue, as even collecting the spectra of a material to confirm its authenticity can potentially expose sensitive information to the inspecting party \cite{glaser_zero-knowledge_2014}. Particularly if non-weapon states are to contribute to treaty verification, it is important to prevent the further dissemination of nuclear weapons knowledge.

\begin{figure}%[htbp!]
\begin{center}
\includegraphics[natwidth=162bp,natheight=227bp, scale=0.45]{./figs/cyclus_interdiscipline.png}
\end{center}
\caption{The \Cyclus nuclear fuel cycle simulator provides a testbed to integrate innovations in treaty verification across many disciplines.}
\label{fig:cyclus_diagram}
\end{figure}

An effective treaty verification regime must synthesize knowledge from the realms of political science, international relations, nuclear physics and engineering, and even behavioral psychology.  Figure \ref{fig:cyclus_diagram} illustrates  the role of a fuel cycle simulator such as \Cyclus in bringing together these disparate fields to provide insights into proposed verification technologies. A fuel cycle simulator tracks the flow of nuclear material through the facilities in a fuel cycle. It creates synthetic data, such as what would be available to an inspector, for many different facilities simultaneously while incorporating a system-level perspective of proliferation scenarios. This synthetic data can then be used as a testbed to investigate the efficacy of new detection and analysis techniques. In this way, simulators can be used to  to illucidate the strengths and weaknesses of various verification strategies.



\section{Factors That Correlate to Pursuit of Nuclear Weapons}
\label{s_factors}

Social science literature has suggested that a variety of political and technical factors that may motivate a state to pursue nuclear weapons. Political factors include: degree to which governing structure is authoritarian versus democratic, level of military spending, degree to which state is isolated militarily, and level of conflict with other states. Technical factors include: degree to which the state's scientific expertise is integrated into the international community, nuclear reactor experience, indigenous reserves of natural uraniaum, and the ability to enrich uranium\cite{hymans_2012}.  \TODO{Hymans, Li et al, Montgomery and Sagan, Sagan, Singh and Way}.

We have compiled a database of information that quantifies each of these eight factors for states at important historical points, publically available on github\footnote{https://github.com/CNERG/historical\_prolif/blob/master/clean\_raw\_data.csv}, along with documentation on source data and assumptions\TODO{README}.  The set includes 42 unique states that have historically had either nuclear energy or weapons technology.  The 24 states that have never pursued weapons have data compiled for 2015. The 19 states that have pursued weapons at some point in the past have data for the year in which they pursued as well as the year in which they acquired a weapon, if applicable. Pursuit and acquisition dates are coded from \TODO{WHAT? Latency set?}.  \TODO{Singh and Way (is this correct?)} define pursuit date as the first year in which a significant decision to pursue nuclear weapons was made such as a political decision by cabinet-level officials, movement toward weaponization, or development of single-use, dedicated technology.  Acquire date indicates the year in which either the first explosion of a nuclear device occurred or the complete assembly of a weapon since not all countries tested their nuclear weapons.

\begin{table}
\centering
\begin{tabular}{|c|c|}
\hline
\textbf{Factor}        & \textbf{Source Database} \\
\hline
Authoritarian            & Center for Systemic Peace \\
                          & Polity IV Annual Time-Series, 1800-2015\cite{polity_scores}\\
\hline
Military Spending & Stockholm International Peace Research Institute \\
    & Military Expenditure Database 1949-2015\cite{mil_sp} \\
\hline
Military Isolation & Rice University \\
& The Alliance Treaty Obligations and Provision Project\cite{mil_iso}\\
\hline
Conflict & ?????\TODO{Fill in and cite} \\
\hline
Scientific Network     & Authors' Expert Opinion\footnote{Considers GDP, opportunities for scientists to study abroad, nuclear infrastructure, technical human capital} \\
\hline
Nuclear Reactors           &  IAEA Power Reactor Database \cite{power_react}\\

                         & IAEA Research Reactor Database \cite{research_react}\\
\hline
Enrichment/Reprocessing   & Nuclear Latency Dataset \cite{fuhrmann_2015}\TODO{Add Fuhrmann} \\
\hline
Uranium Reserves  &    OECD Uranium: Resources, Production and Demand \cite{noauthor_uranium_2014} \\
\hline
\end{tabular}
\caption{Source data for each factor contributing to pursuit of nuclear weapons addressed in this paper.\TODO{SYSTEMIC PEACE HAS COPYRIGHT RESTRICTIONS}}
\label{tab:factor_sources}
\end{table}

\subsection{Pursuit Score}\label{s_pe}
The source used to define each factor has been taken from social science literature and is listed in Table \ref{tab:factor_sources} The raw data for each attribute has been normalized on a 1-10 scale so that all factors can be compared directly, as described in Appendix \ref{s_appendix}. The conflict requires a special note. Conflict has been defined historically for a given state by identifying three significant state-pair relationships.  Each of these relationships is coded as enemy, neutral, or ally.  Each of the two states in the pair is also identified as being a non-weapons state, known to be pursuing weapons, or a weapons-state.  The combination of weapons status and relationship status is combined to provide a conflict score for each pair. The net conflict factor is the average of the state's three conflict scores.

Once every state had been assigned a 0-10 score for each factor, a correlation
analysis was applied to the derived factor scores to quantify the degree to
which each individual factor is correlated to the decision whether or not to
pursue weapons. The resulting weight for each factor is derived from the Pearson correlation coefficient. The bivariate correlation between the factor and the score and can be determined using the equation \ref{eqn:correlation}:
\begin{equation}
    \label{eqn:correlation}
    w_{f} = \frac{\sum_{i=0}^{N} (f_{i} - \bar{f}) (s_{i} - \bar{s})}
                 {\sqrt{\sum_{i=0}^{N}\left(f_{i} - \bar{f}\right)}
                 \sqrt{\sum_{i=0}^{N}\left(s_{i} - \bar{s}\right)}},
\end{equation}
$N$ corresponds to the number of states, $f_{i}$ and $s_{i}$ to the factor and the score of a given state $i$, respectively,  and $\bar{f}$ and $\bar{s}$ to
the mean factor and the mean score over all the states.  The correlation coefficients are then normalized so that the final pursuit score for a state can be defined a weighted linear combination of its different factors. The normalized weights of the 8 factors are shown in Table \ref{tab:factor_weights}. 

\begin{table}
\centering
\begin{tabular}{|c|c|}
\hline
\textbf{Factor}        & \textbf{Weight} \\
\hline
Authoritarian   & 0.12 \\
Military Isolation & 0.075 \\
Reactors           & -0.18 \\
Enrichment \& Reprocessing & 0.10 \\
Scientific Network & 0.05 \\
Military Spending & 0.21 \\
Conflict  & 0.26 \\
Uranium Reserves &  0.0 \\
\hline
\end{tabular}
\caption{Relative weighting of each factor toward pursuit decision as determined by correlation analysis of historical data. Note reactor technology is anti-correlated.}
\label{tab:factor_weights}
\end{table}

Pursuit scores can range between 0 and 10.  Confidence in the weights was gained by applying the weighted equation to the historical data and examining degree to which scores accurately matched historical pursuit decisions.  Historically based scores are shown in Table \ref{tab:state_scores} for the year in which each state explored or pursued a weapon, or 2015 for states that never developed weapons programs. The historical scores are ranked such that states that actually pursued weapons have the highest scores (red).

\begin{table}
  \centering
  \begin{minipage}{.5\textwidth}

\begin{tabular}{|c|c|c|}
\hline
\textbf{State} & \textbf{Year}  & \textbf{Pursuit Score} \\
\hline
USSR & 1945 & \color{red}{8.9} \\
Iran & 1985 & \color{red}{8.3} \\
Iraq & 1983 & \color{red}{8.2} \\
N. Korea & 1980 & \color{red}{7.7} \\
Libya & 1970 & \color{red}{7.4} \\
Egypt & 1965 & \color{red}{7.3} \\
Syria & 2000 & \color{red}{6.9} \\
France & 1954 & \color{red}{6.8} \\
Algeria & 1983 & \color{red}{6.6} \\
Saudi Arabia & 2015 & 6.5 \\
US & 1942 & \color{red}{6.5} \\
India & 1964 & \color{red}{6.4} \\
China & 1955 & \color{red}{6.3} \\
UAE & 2015 & 6.3 \\
Israel & 1960 & \color{red}{6.2} \\
Argentina & 1978 & \color{red}{6.1} \\
S. Africa & 1974 & \color{red}{5.9} \\
UK & 1947 & \color{red}{5.8} \\
Pakistan & 1972 & \color{red}{5.3} \\
Armenia & 2015 & 5.0 \\
Algeria & 2015 & 4.9 \\
Sweden & 1946 & \color{red}{4.8} \\

\hline
\end{tabular}
\end{minipage}\hfill
\begin{minipage}{.5\textwidth}
\begin{tabular}{|c|c|c|}
\hline
\textbf{State} & \textbf{Year}  & \textbf{Pursuit Score} \\
\hline

Romania & 1985 & \color{red}{4.5} \\
Indonesia & 1965 & \color{red}{4.4} \\
Switzerland & 1946 & \color{red}{4.4} \\
Belarus & 2015 & 4.4 \\
S. Korea & 1970 & \color{red}{4.4} \\
Brazil & 1978 & \color{red}{4.3} \\
Australia & 1961 & \color{red}{3.9} \\
Ukraine & 2015 & 3.2 \\
Kazakhstan & 2015 & 3.1 \\
Lithuania & 2015 & 3.1 \\
Japan & 2015 & 3.0 \\
Netherlands & 2015 & 2.7 \\
Finland & 2015 & 2.5 \\
Germany & 2015 & 2.5 \\
Bulgaria & 2015 & 2.4 \\
Mexico & 2015 & 2.0 \\
Slovakia & 2015 & 1.8 \\
Hungary & 2015 & 1.6 \\
Spain & 2015 & 1.6 \\
Czech Republic & 2015 & 1.3 \\
Canada & 2015 & 1.2 \\
Belgium & 2015 & 1.0 \\
%        & & \\
\hline
\end{tabular}
\end{minipage}\hfill
\caption{Historical scores assigned to states based on their factor values at the designated year. Weighting accurately gives high scores to states that acquired (red), pursued or explored (orange), and low scores to those that never investigated a nuclear weapons program (black).}
\label{tab:state_scores}
\end{table}


Two major insights arise from this analysis. Firstly, a state's reactor technology is anti-correlated to pursuing a nuclear weapon.  Denoted in Table \ref{tab:factor_weights} with a minus sign for illustrative purposes, in practice the factor scale was inverted such that maximum reactors led to a minimum reactor score, and then the absolute value of the weight (18\%) was used.  Secondly, indigenous uranium reserves were uncorrelated to weapons programs. These two results were not predicted by the social science literature.



\section{The \Cyclus Fuel Cycle Simulator}
\label{s_methods}

The \gls{CNERG}\footnote{http://cnerg.github.io/} group at the University of Wisconsin has developed the \Cyclus\footnote{http://fuelcycle.org/} nuclear fuel cycle simulator to model all aspects of the nuclear fuel cycle in a flexible way \cite{cyclus_v1_0,cyclus_v1_2,cyclus_v1_3}. \Cyclus produces a database file containing information on the flow of nuclear material through the fuel cycle at each timestep.  The database provides information on facility inventories, material composition, transactions between facilities, and facility build and decommissioning histories, among others.

\Cyclus is designed using an agent-based framework, meaning that each actor  in a fuel cycle (such as a mine, a nuclear reactor, or even a governing body) is modeled as an independent agent \cite{jennings_agent-based_2000, taylor2014agent}.  Each agent in the simulation is self-contained and may include physics, economics, or behavioral components \cite{huff_open_2011,gidden_agent-based_2013,gidden_agent-based_2015}.  The agents interact with one another through the \gls{DRE}, which facilitates the trading of resources and commodities \cite{gidden_agent-based_2014}.  At each timestep, agents can choose to request resources.  Resources are defined using both a quantity (e.g. 1 metric ton), and a quality, such as having a composition of 99.7\% $^{238}U$ and 0.3\% $^{235}U$.  The \gls{DRE} then solicits bids from any facilities that are interested in offering those resources. Resources can be offered as bids even if they do not exactly match the requested material. For example, a reactor might request a commodity called ``fuel'', which it has defined as being \gls{UOX}.  It may receive two bids for ``fuel'' that are specified as \gls{UOX} and \gls{MOX}, having two distinct isotopic compositions. After the bids are received, the requestor is able to state a preference for one bid over another. Finally, once the preferences have been applied, the \gls{DRE} calculates all potential trades across all agents, then executes an optimization algorithm to find the solution that most closely fulfills a maximum of requests. It is possible that as a result of the preference adjustment, no trades are executed for some facilities in a given time step.  Once this solution is found, material is transfered across the facilities and the timestep is concluded.

Although \Cyclus was designed to assess the transition from once-through fuel cycles to alternative next-generation scenarios including technologies such as spent fuel recycling, it is also an excellent tool for examining proliferation issues.  Other fuel cycle simulators designed to study energy issues are typically too inflexible to be used for other purposes. For example, they may have extremely accurate physics models of reactor burnup, but no ability to vary or modify other facilities in the fuel cycle.  \Cyclus has three key features that make it well-suited to non-proliferation studies: it is \textit{agent-based}, it tracks \textit{discrete materials}, and it incorporates \textit{social and behavioral interaction models}. Agent-based design allows for modular simulations where individual facilities can be swapped compared in otherwise identical simulations. \Cyclus tracks discrete material flow through the simulation, and uses data from PyNE,\footnote{http://pyne.io/} to track decay and transumutation data at all timesteps \cite{Scopatz2012b, huff_integrated:_2013}. Finally, behavioral modeling allows facilities and institutions to engage in dynamic decision-making based on their preferences, needs, or political constraints.  A specific agent might have preferences based on material composition, physical proximity between facilities, or allowed and disallowed trading partners, which are implemented in a region-institution-facility hierarchy that enables economic modeling \cite{oliver_geniusv2:_2009}.

Behavior is a critical aspect of non-proliferation modeling. For example, an enrichment facility receiving illicit requests for \gls{HEU} may define its own criteria for whether or not to fulfill this order.  It may disallow production of enrichments above a certain level, choose to trade only with specific facilities, or choose to preferentially fulfill requests at one enrichment level over an other.  Likewise, a requestor of \gls{HEU} can make requests at regular or random intervals, and may request randomly or gaussian distributions of material quantity.  
At the insitution level, \Cyclus allows trading decisions between facilities to be controlled by the owner of those facilities, which may be a commercial entity or a nation-state.  This facilitiates modeling of the interactions between multiple states within a region.  






\section{Discussion}
\label{s_dis}

\Cyclus is able to model signatures of diversion from a diverse set of facilities in the nuclear fuel cycle and with a variety of data modalities. Table \ref{tab:modalities} lists a variety of signal modalities and their applications in the fuel cycle (TODO: synonym for schema?).  One modality with diverse set of potential applications is satellite imagery.  We are now developing the software infrastructure to create synthetic satellite images that may contain signals of diversion. Satellite imagery has a variety of applications: tracking personnel or truck movement patterns, thermal or visible signatures of effluent or heat, or major facility changes such as new or removed buildings.

%% TODO: MAKE TABLE WITH MODALITIES AND THEIR APPLICATIONS
% Simulation parameters in RS_3sink.xml at
%/Users/mbmcgarry/git/data_analysis/data/v1.2/random_sink/
\begin{table}
\centering
\begin{tabular}{|c|c|c|}
\hline
\textbf{General}    & Duration (months)       & 100  \\
\textbf{Simulation} & Natural U (\% $^{235}U$) & 0.7  \\
\textbf{Parameters} & LEU (\% $^{235}U$)       & 4.0  \\
                    & HEU (\% $^{235}U$)       & 90.0 \\
\hline
\textbf{Enrichment} & SWU Capacity (kg-SWU/month) & 180  \\
\textbf{Facility}   & Tails Assay (\% $^{235}U$)   & 0.3  \\
\hline
\textbf{LEU Demand} & Mean Qty (kg)       & 33.0  \\
                    & $\sigma$ (kg)       & 0.5  \\
\hline
\textbf{HEU Demand} & Qty (kg)            & 0.03  \\
                    & Avg Rate of Occurrence & 1/5 \\ 
\hline
\end{tabular}
\caption{Simulation parameters for \gls{HEU} diversion scenario.}
\label{tab:sim_params}
\end{table}

These diverse datasets can be combined to highlight signatures of diversion that are small enough to be hidden in the noise of individual signals.  We have illustrated this technique by combining time-series data for power consumption and declared \gls{LEU} production for a simple scenario of \gls{HEU} production in an enrichment facility.  More realistic scenarios require advanced anomaly detection techniques such as those being developed at \gls{UM}. A collaboration with \gls{UM} and \gls{Sandia} will investigate ways to optimize subsets of diverse signal modalities to ensure reliable detection while minimizing resource usage.

The \Cyclus fuel cycle simulator is being used as a framework for combining techniques and knowledge from a variety of disciplines to support a cohesive approach to treaty verification.  Moving forward, \Cyclus will be used to study more complex and realistic diversion scenarios.  Additionally, \Cyclus has the capability to produce synthetic signals of inherent physical processes such as neutron spectra of various materials.  In this way, \Cyclus simulations can provide theoretical signals to researchers developing experimental detectors in order to test sensitivity and detector response.  \Cyclus is also being used to explore behavioral mode ***


Probabilistic models for behavior based on the actor's risk-perception will be explored.  Ongoing collaborations as part of the \gls{CVT} are examining the mechanisms and limits of expanding anomaly detection algorithms with other types of data, such as social media chatter.  Due to the inherently interdisciplinary nature of this work, new external collaborations are sought with experts in behavioral modeling. Innovative ideas on detection modalities and diversion detection techniques are also welcomed.



\textit{This work was funded in-part by the Consortium for Verification Technology under Department of Energy National Nuclear Security Administration award number DE-NA0002534”}

\section{Appendix: Factor Conversion Tables}
\label{s_appendix}

%\iffalse

%\begin{landscape}
\begin{table}
\centering
\begin{tabular}{|c|c|c|c|c|c|c|c|c|c|}
\hline
\textbf{Factor}      & \textbf{Auth}          & \textbf{Enrich/}      & \multicolumn{2}{c|}{\textbf{Military Iso.}}          & \textbf{Mil. Spend} &  \textbf{Reactors}  & \textbf{Sci.} & \textbf{Uran.} \\
\textbf{Score}  &                        &  \textbf{Repro.}   & \multicolumn{2}{c|}{$10 - (A_{NNWS}+A_{NWS})$}  & \textbf{(\%GDP)}    &   (Power+Research) & \textbf{Net.} &  \textbf{Res} \\
\cline{4-5}
 (FS)           &                         &                       &  NNWS                 & NWS               &                    & $10 - R_{all}$&               &              \\


\hline
\textbf{0}      &  0                     &  0                   &  --                        &  --                     &  --                 &   0                   &    --         &  0           \\   
\textbf{1}      &  1                     &  --                  &  1-2                       &  --                     &  $<$ 1              &   1-3 planned         &    --         &  --          \\
\textbf{2}      &  2                     &  --                  &  3-4                       &  --                     &  [1, 2)             &   4+ planned          &    --         &  --          \\
\textbf{3}      &  3                     &  --                  &  5+                        &  --                     &  --                 &    --                 &    --         &  --          \\
\textbf{4}      &  4                     &  --                  &  --                        &  --                     &  [2, 3)             &    1-3 built          &     1         &  --          \\
\textbf{5}      &  5                     &  --                  &  --                        &  1                      &  --                 &       --              &    --         &  --          \\
\textbf{6}      &  6                     &  --                  &  --                        &  2                      &  --                 &       --              &    --         &  --          \\
\textbf{7}      &  7                     &  --                  &  --                        &  3+                     &  [3, 5)             &    4-7 built          &     2         &  --          \\
\textbf{8}      &  8                     &  --                  &  --                        &  --                     &  --                 &       --              &    --         &  --          \\
\textbf{9}      &  9                     &  --                  &  --                        &  --                     &  --                 &       --              &    --         &  --          \\
\textbf{10}     &  10                    &  1                   &  --                        &  --                     &   5.0+              &    8+ built           &     3         &  10          \\

\hline
\end{tabular}
\caption{Conversion table from raw data to final factor score (FS). Square brackets are inclusive, parentheses are exclusive, such that [1,2) indicates $1<=x<2$. Reactors and military alliances (used to define military isolation) are both anti-correlated to pursuit so the final factor score for these factors is 10 minus the value shown in the table. Conflict factor is defined separately in table \ref{tab:conflict}. }
\label{tab:factor_conversions}
\end{table}
%\end{landscape}


\begin{table}
\centering
\begin{tabular}{|c|c|c|c|}
\hline
\textbf{Nuc. Weapon Status} & \textbf{Allies}  & \textbf{Neutral}  & \textbf{Enemies} \\
\hline
NNWS - NNWS     & 2 & 2 & 6 \\
NNWS - Pursue   & 3 & 4 & 8 \\
NNWS - NWS      & 1 & 4 & 7 \\
Pursue - Pursue & 4 & 5 & 9 \\
Pursue - NWS    & 3 & 6 & 10 \\
NWS - NWS       & 1 & 3 & 5 \\
\hline
\end{tabular}
\caption{\TODO{caption}}
\label{tab:conflict}
\end{table}
%\end{landscape}


%%%%%%%%%%%%%%%%%%%%%%%%%%%%%%%%%%%%%%%%%%%%%%%%%%%%%%%%%%%%%%%%%%%%%%%%%%%%%%%%
\begin{small}
\bibliographystyle{ANSurl}
\bibliography{../zotero_160516,../zotero_adds_160603,../manual_fixes,../websites_manual}
\end{small}
\end{document}
