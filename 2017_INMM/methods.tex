\section{A \Cyclus Model of Nuclear Weapon Pursuit}
\label{s_methods}

We have applied the factors correlated to pursuit of nuclear weapons in a regional model of state interactions using \Cyclus.  The \gls{CNERG}\footnote{http://cnerg.github.io/} group at the University of Wisconsin has developed the \Cyclus\footnote{http://fuelcycle.org/} nuclear fuel cycle simulator to model all aspects of the nuclear fuel cycle in a flexible way.  \Cyclus has three key features: it is \textit{agent-based}, it tracks \textit{discrete materials}, and it incorporates \textit{social and behavioral interaction models}\cite{jennings_agent-based_2000, gidden_agent-based_2013, taylor2014agent}. This design allows customized facilities and institutions to engage in dynamic decision-making based on their preferences, needs, or political constraints across a wide range of scenarios.A region-institution-facility framework captures have preferences based on material composition, physical proximity between facilities, or preferred trading partners.

\subsection{The Forward Model}

The region-institution-facility design has been used to develop the \gls{NWPM}. This forward model features two custom archetypes, an interaction region and a state institution\cite{mbmore}.  The state institution represents a nation-state and includes time-dynamic information about each of the pursuit factors. The interaction region is an omniscient presence in the simulation that tracks weapon status as well as interactive pursuit factors such as conflict (as described in section \ref{s_pe}), and communicates that factor to each individual state. 

Each of the motivating factors is defined for every state in a time dynamic way. Individual factors must have values between 0-10. There are several time dynamic parameterizations currently available in the model: constant, linear growth or decline, step-function (at either a specified or a randomly chosen time), or power-law.  These functions enable modeling of characteristics such as growth in military spending, development of new technologies (such as enrichment), and sudden changes to factors such as governing structure or inter-state conflict.  At each timestep, the state institution combines all of these factor values into a pursuit score using the weighted linear equation defined in section \ref{s_factors}.

 \subsection{Likelihood}

\begin{figure}%[htbp!]
%\begin{center}
\includegraphics[scale=0.7]{./figs/pe_likely.png}
%\end{center}
\caption{Fraction of states with any nuclear technology (dataset from section \ref{s_factors}) that pursued weapons at some point in the last 75 years, given their pursuit scores. Although a power-law curve (black) and a weighted linear fit (red) are equivalent with the existing data, a complete set of nation-states would increase the relative weight of lower scores, making the exponential curve a better fit.}
\label{fig:likely}
\end{figure}
 
The pursuit score ($s$) is then converted into a likelihood that the state will pursue a weapon. The relationship between pursuit score and likelihood of pursuing a weapon has been characterized using historical data of the 43 states that have developed nuclear technology since 1942. Figure \ref{fig:likely} shows the fraction of states with a given pursuit score that pursued a weapon at some point between 1942 and 2015. A power-law (red) or a linear fit (black) are  equivalently valid with the 43 state dataset. However, a global dataset of nation-states will have disproportionately lower scores and zero additional incidences of pursuit, so we consider the power-law paramaterization to be more representative.

While the historical data provides the probability of pursuit integrated over 75 years ($T$), a single-year probability is needed for the forward model. A state can only choose to pursue in a given year ($p$) if it is not already pursuing, therefore probability must be defined by the absence of pursuit. The probability that a state will Not pursue in a single years is $1 - p$. The likelihood of non-pursuit $\bar{L}$ integrated over $N$ years is:
\begin{equation}
\bar{L} = (1- p)^{N}
\end{equation}
And the likelihood of a pursuit ($L$) integrated over $N$ years becomes:
\begin{equation}
L = 1 - (1-p)^{N}
  \end{equation}
Therefore the probability of pursuit in a single year is:
\begin{equation}
p = 1 - (1 - L)^{1.0/N}
\label{eqn:likely_eqn}
\end{equation}

In the \gls{NWPM} forward model, each state's score is converted to a single year probability of pursuit using the power-law fit shown in Figure \ref{fig:likely}. The actual conversion is bounded using two Heavyside functions such that scores below a lower threshold (e.g. 4) are forced to a likelihood of zero, while scores above an upper threshold (e.g. 9) are forced to the value of the score at that threshold. At each time in the simulation, a random number generator is queried using the score-derived probability ($p$) to determine whether or not the pursuit event occurs. If a model is provided for the likelihood of acquisition given pursuit, then acquistion is also tracked. Both pursuit and acquisition decisions then influence future conflict relationships with the other states.

%If the determination is for pursuit to occur, the state deploys a secret enrichment facility. If a state is already pursuing a weapon, then the model determines whether it succeeds in acquiring a weapon at that timestep using a median time to acquire of 7.5 years, based on historical data for states that have acquired weapons. On the timestep in which a state succeeds in acquiring a weapon, highly enriched uranium is shipped out of the secret enrichment facility.

