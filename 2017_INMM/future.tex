\section{Summary and Future Work}
\label{s_future}

We have developed a forward model that characterizes the probability that a state will pursue a nuclear weapons progam based on a set of eight socio-political and technical factors: governing structure, reactor technology,  inter-state conflict,  enrichment technology, military isolation, military spending, scientific network, and indigenous uranium reserves. To develop this model, we first assembled a historical database of quantitative data to characterize the factors, and then used the database to determine the degree to which each factor is correlated to a pursuit decision. Historical data was also used to inform the conversion between the pursuit score and a likelihood of developing a weapons program. 

Our analysis yielded unexpected results for two of the eight factors suggested by political science literature as motivations for pursuing weapons. The presence of uranium reserves was uncorrelated to a pursuit decision, suggesting that access to natural uranium is not a bottleneck.  Perhaps more surprising, access to reactor technology was anti-correlated with a pursuit decision, meaning the more reactors a state had, the less likely it was to pursue. It is possible that this effect is a consequence of the existing nonproliferation regime - states that make the strongest nonproliferation commitments are rewarded with assistance in developing nuclear energy programs.  It would be instructive to investigate both of these factors further, for example, to see whether there is a distinction in the correlation of research versus commercial reactors to a pursuit decision.

It is important to note that this model is meant to be indicative rather than predictive. Human behavior is fundamentally difficult to explain, let alone model.  The actions of individuals can have outsized effects, for example A.Q. Khan's role in Pakistan's weapons program.  Humans do not always act rationally, and groups can make very different decisions from individuals.  Expertise from psychology, game theory, sociology and international relations could certainly improve our model. Ultimately though,there will always be some level of unpredictability in human behavior that prevents models such as this from being predictive.

Nonetheless, we believe this model can be used to offer valuable insights into the relative risks of potential future nonproliferation paradigms. While we cannot predict whether or not State A will pursue a weapon in the next 10 years, we can investigate ways to reduce the risk of that pursuit occurring based on our understanding of these factors. For example, we are currently applying the model to scenarios considering the creation of regional multilateral enrichment facilities as a means of reducing the spread of enrichment technology.  Would a regional multilateral enrichment agreement in the Middle East reduce proliferation risk under all possible conflict scenarios, or are there potential landscapes in which risk would not be further mitigated by such an approach?  We anticipate that the model will also be useful in examining the impacts of other future proposed treaties or international agreements.

\textit{This work was funded by the Consortium for Verification Technology under Department of Energy National Nuclear Security Administration award number DE-NA0002534”}
