\section{Discussion and Future Work}
\label{s_dis}

It is important to note that this model is meant to be indicative rather than predictive. Human behavior is fundamentally difficult to explain, let alone model.  The actions of individuals can have outsized effects, for example A.Q. Khan's role in Packistan's weapons program.  Humans do not always act rationally, and groups can make very different decisions from individuals.  Expertise from psychology, game theory, sociology and international relations could certainly improve our model but there will always be some level of unpredictability in human behavior that prevents models such as this from being predictive.

Nonetheless, we believe this model can be used to offer valuable insights into the relative risks of proposed future paradigms. While we cannot predict whether or not State A will pursue a weapon in the next 10 years, we can investigate ways to reduce the risk of that pursuit occurring based on our understanding of these factors. For example, we plan to apply the model to scenarios investigating how regional multilateral enrichment facilities would affect proliferation risk on a regional or global scale.  \TODO{ANY OTHER IDEAS FOR FUTURE WORK? PERHAPS HEDGING STRATEGIES?}


\textit{This work was funded by the Consortium for Verification Technology under Department of Energy National Nuclear Security Administration award number DE-NA0002534”}
