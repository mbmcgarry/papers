%\documentclass{proc}  % 2-column format
\documentclass[12pt]{paper}
%\documentclass{ntmanuscript}
%\documentclass[review]{elsarticle}
\usepackage{mathptmx} % Nearly Times New Roman
\usepackage[acronym,toc]{glossaries}
\newacronym{UW}{UW}{University of Wisconsin}
\newacronym{UM}{UM}{University of Michigan}
\newacronym{US}{US}{United States}
\newacronym{HEU}{HEU}{highly enriched uranium}
\newacronym{LEU}{LEU}{low enriched uranium}
\newacronym{U}{U}{uranium}
\newacronym{U235}{U-235}{uranium 235}
\newacronym{U238}{U-238}{uranium 238}
\newacronym{UF6}{UF6}{uranium hexafluoride}
\newacronym{SWU}{SWU}{separative work unit}
\newacronym{CNERG}{CNERG}{Computational Nuclear Engineering Research Group}
\newacronym{WGP}{WGP}{weapons-grade plutonium}
\newacronym{NPT}{NPT}{Nuclear Nonproliferation Treaty}
\newacronym{IAEA}{IAEA}{International Atomic Energy Agency}
\newacronym{JCPOA}{JCPOA}{Joint Comprehensive Plan of Action}
\newacronym{ISIS}{ISIS}{Institute for Science and International Security}
\newacronym{CVT}{CVT}{Consortium for Verification Technology}
%\newacronym{<++>}{<++>}{<++>}
%\newacronym{<++>}{<++>}{<++>}
%\newacronym{<++>}{<++>}{<++>}
%\newacronym{<++>}{<++>}{<++>}

%\makeglossaries
%%%%%%%%%%%%%%%%%%%%%%%%%%%%%%%%%%%

\usepackage{color}
\usepackage{subcaption}
\usepackage{graphicx}
\usepackage{booktabs} % nice rules for tables
\usepackage{microtype} % if using PDF
\usepackage{xspace}
\usepackage{listings}
\usepackage{textcomp}
\usepackage{multicol,tabularx,capt-of}
\usepackage{multirow}
%\usepackage{ulem}
\usepackage{pdflscape}
% Page length commands go here in the preamble
\setlength{\oddsidemargin}{-0.25in} % Left margin of 1 in + 0 in = 1 in
\setlength{\textwidth}{7in}   % Right margin of 8.5 in - 1 in - 6.5 in = 1 in
\setlength{\topmargin}{-.75in}  % Top margin of 2 in -0.75 in = 1 in
\setlength{\textheight}{9.2in}  % Lower margin of 11 in - 9 in - 1 in = 1 in



\definecolor{listinggray}{gray}{0.9}
\definecolor{lbcolor}{rgb}{0.9,0.9,0.9}
\definecolor{burgundy}{rgb}{0.5, 0.0, 0.13}
\definecolor{burntorange}{rgb}{0.8, 0.33, 0.0}
\definecolor{chromeyellow}{rgb}{1.0, 0.65, 0.0}
\definecolor{darkred}{rgb}{0.55, 0.0, 0.0}

\lstset{
    %backgroundcolor=\color{lbcolor},
    language={C++},
    tabsize=4,
    rulecolor=\color{black},
    upquote=true,
    aboveskip={1.5\baselineskip},
    belowskip={1.5\baselineskip},
    columns=fixed,
    extendedchars=true,
    breaklines=true,
    prebreak=\raisebox{0ex}[0ex][0ex]{\ensuremath{\hookleftarrow}},
    frame=single,
    showtabs=false,
    showspaces=false,
    showstringspaces=false,
    basicstyle=\scriptsize\ttfamily\color{green!40!black},
    keywordstyle=\color[rgb]{0,0,1.0},
    commentstyle=\color[rgb]{0.133,0.545,0.133},
    stringstyle=\color[rgb]{0.627,0.126,0.941},
    numberstyle=\color[rgb]{0,1,0},
    identifierstyle=\color{black},
    captionpos=t,
}

\newcommand{\code}[1]{\lstinline[basicstyle=\ttfamily\color{green!40!black}]|#1|}
\newcommand{\units}[1] {\:\text{#1}}%
\newcommand{\SN}{S$_N$}
\newcommand{\cyclus}{\textsc{Cyclus}\xspace}
\newcommand{\Cyclus}{\cyclus}
\newcommand{\citeme}{\textcolor{red}{CITE}\xspace}
\newcommand{\TODO}[1] {{\color{red}\textbf{TODO: #1}}}%

\newcommand{\comment}[1]{{\color{green}\textbf{#1}}}

%%%%%%%%%%%%%%%%%%%%%%%%%%%%%%%%%%%
\begin{document}


%\begin{frontmatter}
\title{Fundamentals of Centrifuge Enrichment: A Technical Primer}

% Authors. Separated by commas
\author{
  Meghan B. McGarry$^1$,
  Drew Buys$^1$,
  Paul P.H. Wilson$^1$}


\date{}
% Institutes of the authors
\institution{$^1$Department of Nuclear Engineering and Engineering Physics,
University of Wisconsin-Madison}
%\\$^2$Sandia National Laboratories}
% Information concerning the person submitting the manuscript
%\submitter{Meghan B. McGarry}
%\submitteraddress{1500 Engineering Drive, Madison, WI, USA}
%\submitteremail{mbmcgarry@wisc.edu}

% No more than three keywords, though each can be a phrase
%\keywords{fuel cycle, simulatiom, non-proliferation}
\maketitle

% I. Motivation
% x   -  Identify Factors that motivate
% x  II. Benchmark model against historical data:
% x A. Develop historical database
% x   - What sources?
% x   - What model to convert to 10pt scale?
% x        - Conflict
% x B. Determine relative weighting of factors
% x   - Calculate values for historical database
% x   - PCA to determine relative weights
% ** C. Table of State Score Results
% III. Develop forward model 
%   - From score to a likelihood, using historical data
% IV. Limitations of model
%   - small dataset
%   - threshold value for proliferation
%   - no good model for scientific network
% V. Future work
%   - Apply to case study (JCPOA?)
% VI Appendix of factor conversions

\begin{abstract}

 

\end{abstract}


%\end{frontmatter}


\section{Introduction}
\label{s_intro}
Under the \gls{JCPOA}, colloquially known as the Iran deal, Iran has accepted limits on its nuclear program in order to ensure the world that it is not developing nuclear weapons. Chief among these limits are restrictions on Iran’s ability to enrich uranium through its centrifuge program. Centrifuge technology is the focus of the agreement because it is the turnkey technology in developing nuclear weapons. Without a capable centrifuge enrichment program, a potential proliferant country has a much more difficult path to arm itself.

Prior to the implementation of the \gls{JCPOA}, Iran had more than 10,000 centrifuges in more than 70 cascades. The most contentious negotiations in the lead-up to agreeing to the \gls{JCPOA} surrounded the type and number of centrifuges that would be available to Iran during the agreement. It is now operating only 5,060 of its most rudimentary centrifuges (the IR-1), configured in 30 cascades, at the Natanz Fuel Enrichment Plant. The centrifuge engineering constraints in the \gls{JCPOA} limit the amount of enriched uranium Iran can produce over the course of the agreement.

The \gls{JCPOA} was designed to ensure a minimum breakout time of one year. There is no way for Iran to put more uranium in its existing centrifuges and make a bomb more quickly. To acquire a significant quantity of weapons grade uranium more quickly, Iran would either have to secretly build more centrifuges or remove them from storage facilities that are now under surveillance by the \gls{IAEA}.


\section{Nuclear Nonproliferation Basics}
To understand the policy implications of the Iran deal, it’s necessary to understand the terminology and framework in which it is discussed. Naturally occurring uranium—or natural uranium— consists of two isotopes, \gls{U235} and \gls{U238}. Natural uranium is only 0.7\% \gls{U235} by weight. A nuclear weapon requires at least 90\% \gls{U235}. Below this level of enrichment, a nuclear reaction cannot be sustained to produce an explosion.  Centrifugation is the most common way to enrich the material, increasing the concentration of \gls{U235}.

It is important to note that building a nuclear weapon is not the only reason to enrich uranium. Enriched uranium is necessary to fuel many types of civilian nuclear power reactors, and is also used in some medical and research applications. Iran claims to use its centrifuge enrichment program to create 3.5\% \gls{LEU} for nuclear power plants and 19.75\% enriched uranium, also \gls{LEU}, for medical purposes. Uranium enriched past the 20\% level, including weapons-grade uranium, is considered \gls{HEU}.

The \gls{JCPOA} is designed to hold Iran to these claims by limiting both the level to which Iran can enrich uranium and the total amount of enriched uranium allowed in the country. It takes as little as 9 kg of 90\% enriched uranium to create a sophisticated nuclear weapon. Simpler weapons require up to 50 kg\cite{ucs_2009}. %(Union of Concerned Scientists 2009)

A gas centrifuge used for uranium enrichment is much like any other centrifuge. Uranium is enriched in centrifuges in its gaseous form as \gls{UF6} , but enrichment levels and outputs are discussed in terms of uranium because that is the form in which it fuels a power plant or arms a bomb.  A centrifuge uses the difference in weight between \gls{U235} and \gls{U238} to separate out the \gls{U235} by spinning it at high velocity. The lighter \gls{U235} collects at the top of the centrifuge while the heavier \gls{U238} sinks to the bottom and is removed. A centrifuge’s ability to complete this task is its separative capacity, the energy required to enrich the uranium in the system to the desired level of enrichment. This is typically given in kilogram \gls{SWU} on uranium per unit time, which is denoted as just SWU. This measurement is the key measure of the effectiveness of a centrifuge and, consequently, was an important parameter in negotiating the \gls{JCPOA}.  


\section{Centrifuge Design}
\label{s_centrifuge}
A gas centrifuge used for uranium enrichment is much like any other centrifuge. It uses the difference in weight between two isotopes of uranium, \gls{U235} and \gls{U238} to separate out the desired isotope, \gls{U235} by spinning it at high velocity. The lighter \gls{U235} collects at the top of the centrifuge while the heavier \gls{U238} sinks to the bottom.

\iffalse
\begin{figure}%[htbp!]
%\begin{center}
\includegraphics[scale=0.7]{./figs/centrifuge.png}
%\end{center}
\caption{A Nuclear Regulatory Commission Diagram of a Simple Gas Centrifuge}
\label{fig:likely}
\end{figure}
\fi

The type of centrifuge used in uranium enrichment is almost always the counter-current gas centrifuge, a drawing of which can be seen in Figure \ref{fig:centrifuge}. Although there are other types, such as the concurrent centrifuge and evaporative centrifuge, the counter-current centrifuge is the only type prominently used in enrichment operations in Iran and Pakistan. The counter-current centrifuge is both more efficient in terms the time it takes to separate uranium and in terms of the energy it uses to do so.

A centrifuge is comprised of several constituent parts. The first thing the outside observer will notice is the large cylindrical case that houses the rest of the centrifuge. Next to draw the eye is the piping running between cases, one of which allows uranium hexafluoride to be fed into the machine as well as separate pipes for removing the product stream containing the more enriched uranium and the tails stream which contains the less enriched or depleted uranium. These pipes run through the case and into the rotor, a cylindrical tube closed on each end with cap pieces, where the separation of isotopes actually happens. Consequently, the rotor is the part which most concerns those working on non-proliferation. Its height, diameter, and the speed at which it can rotate are all important factors in calculating the \gls{SWU} of a centrifuge. 

Inside the rotor, at the very center both vertically and horizontally, is the center post where uranium enters the machine\cite{Olander_1981} %\The Theory of Uranium Centrifuge Enrichment Olander, Donald. 17 March 1981. Progress in Nuclear Energy.
At the top and bottom of the rotor are scoops which help remove separated gas from the centrifuge, depleted gas with a higher concentration of \gls{U238} from the bottom and enriched gas with a higher concentration of \gls{U235} at the top. The bottom scoop also contributes to the counter-current flow within the machine. Between the top scoop and the main chamber of the centrifuge is a device called a baffle which keeps the top scoop from interfering in the flow of the machine . Some centrifuges include a part called a bellows that connects rotor sections. This is not a bellows in the way one might imagine a medieval blacksmith using one, but a connector made of steel or carbon fiber that allows the rotor greater flexibility when being rotated at high speed.

Centrifuges must be constructed with materials that are able to withstand both the high speed at which they are rotated and the corrosive effects of UF6. These materials include aluminum alloys, maraging steel, carbon fiber and other synthetics\cite{shin_sung} % Implications of Iran's Success in Developing Fourth-Generation Centrifuge Technology Shin Sung Tack. Online Series KINU
. The typical progression in development for countries building their own centrifuges is to begin with aluminum, move to maraging steel and then to carbon fiber. There are several reasons for this. The first is that aluminum is cheaper and easier to obtain than either maraging steel or carbon fiber. This makes it better for a proof of concept centrifuge design and simpler to begin with if a state is concerned about export controls limiting their access to materials. One of the consistent barriers for Iran's nuclear program has been its inability to buy quality carbon fiber or maraging steel abroad due to international sanctions and not having the industry at home to manufacture them at a high enough quality. Aluminum is also easier to machine than either maraging steel or carbon fiber. According to the \gls{ISIS}, this has led to Iranian engineers using aluminum end caps on steel rotor tubes as a way to preserve the steel and carbon fiber they do have\cite{isis_webpage} %Iran's new centrifuge: What do we know about it? David Albright, Jacqueline Shire, Paul Brannan. April 13, 2010. Institute for Science and International Security..  

\section{Centrifuge Equations}


\begin{multline}
  \label{eqn:raetz}
  \delta U_{Raetz}(L,F,\theta,Z_{P})
  =\frac{1}{2}F\theta (1-\theta)
  \left ( \frac{\Delta M}{2RT}v_{a}^{2} \right )^{2}
  \left ( \frac{r_{2}}{a} \right )^{4}
  \left [1-\left ( \frac{r_{1}}{r_2} \right )^{2}
  \right ] ^{2} \\
* \left[
  \left( \frac{1+L/F}{\theta} \right )
  \left ( 1 - e^{-A_{p}(L,F,\theta)Z_{p}}\right )
  + \left (\frac{L/F}{1-\theta}\right )
  \left \{ 1-e^{-A_{W}(L,F,\theta)(Z - Z_{P})} \right \} 
  \right]^{2}
\end{multline}


\begin{equation}
  A_{P} = \left (
  \frac{2 \pi D_{\rho}}{ln(r_{2}/r_{1})}
  \right )
  \left (
  \frac{\theta/F}{(1+L/F)(1-\theta+L/F)}
  \right )
\end{equation}
\begin{equation}
  A_{W} = \left (
  \frac{2 \pi D_{\rho}}{ln(r_{2}/r_{1})}
  \right )
  \left (
  \frac{\theta/F}{(L/F)(1-\theta+L/F)}
  \right )
\end{equation}

\begin{equation}
  Z_{P} = \left (
  \frac{(1-\theta)(1+L/F)}{1-\theta+L/F}
  \right ) Z
\end{equation}


F = Feed flow in moles/second
P = Product flow in moles/second
$\theta$ =P/F  (cut)
$\Delta$ M=Difference in molar mass between $U_{235}F_{6}$  and  $U_{238}F_{6}$
R=universal gas constant
T=Average temperature of the gas
$v_{a}^{2}$=peripheral velocity of the rotor in meters per second
$r_{2}$=radius at which product is extracted
a=radius of rotor
$r_{1}$=radius at which waste is extracted
L=countercurrent flow rate
$D_{\rho}$ = gas diffusion constant
Z = rotor height

\begin{equation}
  \label{eqn:sep_fac}
\alpha = \sqrt{\frac{2\delta U}{\theta/(1-\theta)L}} + 1
\end{equation}

One important note for the reader: Equation \ref{eqn:raetz} is typically discussed in the literature as being in terms of molar fraction or moles of \gls{UF6} per second. Equation \ref{eqn:sep_fac}, and most of the equations in the cascade section, are in terms of atom fraction or kilograms of uranium per year. This is a discrepancy found both in the technical literature and in the non-proliferation policy world. For example, Iran often reports its \gls{SWU} capacity in terms of enriched kilograms of \gls{UF6}. In the west when Iran’s enrichment limits are in terms of kilograms of enriched uranium – because that’s the mass that counts against the limits in the \gls{JCPOA} and the 90kg necessary to build a nuclear weapon.




\section{Iran's Centrifuge Program}
\label{s_iran}

Iran currently has seven centrifuge designs under development that we know about: the IR-1, IR-2, IR-2m, IR-4, IR-5, IR-6 and IR-8. They are named IR, for Iran, and then a number, which often but doesn’t always correspond with their development order. The simplest and least effective design, the IR-1, is what Iran is able to use under the \gls{JCPOA}. This centrifuge has an effective rotor height of 1.8 meters, a diameter of 10 centimeters and is constructed primarily of aluminum. It can spin at about 335 meters per second and has a projected SWU of just under 1 per year. Iran’s most effective centrifuge design, and the most advanced centrifuge on which it is allowed to do research during the Iran deal, is the IR-8. Little is publically known about its design. A September 2014 \gls{ISIS} report suggests that it is 2.3 meters tall, spins at 550 meters per second and has a projected SWU capacity of 16 per year. Below is a table which compares information on several Iranian centrifuges with similar Pakistani centrifuges that the early Iranian designs were based on\cite{maccalman_2016,isis_website, glaser_?, tack_?,fars_?}.%MacCalman, Molly. "A.Q. Khan Nuclear Smuggling Network." Journal of Strategic Security 9, no. 1 (2016): 104-118

\begin{table}
\centering
\begin{tabular}{|c||c|c|c|c|c|c|c|}
\hline
\textbf{Name} & \textbf{Rotor Height} & \textbf{Diameter} & \textbf{Velocity} & \textbf{Construction} & \textbf{Reported} & \textbf{Efficiency} & \textbf{P-type} \\
     & \textbf{(m)}          &  \textbf{(cm)}    & \textbf{(m/s)}    &              & \textbf{SWU/yr}   &            &        \\
\hline
\textbf{IR-1}  & 1.8 & 10   & 335 & Aluminum       & $\sim$0.75 & 0.49  & P1 \\
\hline
\textbf{IR-2}  & 0.5 & N/A  & N/A & Carbon Fiber   & $\sim$5    & 0.43  & P2 \\
\hline
\textbf{IR-2m} & 1   & 14.5 & 550 & Carbon Fiber   & 5          & N/A   & P2 \\
\hline
\textbf{IR-4}  & 1   & 14.5 & 550 & Carbon Fiber   & $\sim$5    & N/A   & P2 \\
\hline
\textbf{IR-8}  & 2.3 & N/A  & 550 & N/A            & 16         & N/A   & N/A \\
\hline
\textbf{P1}    & 2   & 10   & N/A & Aluminum       & 3          & 0.564 & --- \\
\hline
\textbf{P2}    & 1   & 15   & N/A & Maraging Steel & 6          & 0.465 & --- \\
\hline
\textbf{P3}    & 2   & 20   & N/A & Maraging Steel & 12         & N/A   & --- \\
\hline
\end{tabular}
\caption{A Selection of Iranian Centrifuges and their Pakistani Comparisons. N/A is not available; --- indicates not relevant}
\label{tab:factor_sources}
\end{table}


\section{Cascades}
\label{s_cascade}
A single centrifuge is limited in both the amount of uranium it can enrich and the level to which it can enrich that uranium by its separation coefficient. This limit means that an individual centrifuge cannot develop enough fuel for either a power plant or a bomb. So, in order to enrich uranium at a useful scale centrifuges are set up in cascades. A cascade is made up of groups of centrifuges split into stages that are organized by level of uranium enrichment; each stage is a set of identical centrifuges that enrich fuel from the previous stage. The cascades in each stage are connected in parallel, while the connections between stages are in series . This means that gas in centrifuges in the same stage 
A picture of a cascade would show equal sized rows and columns of centrifuges with uniform spacing between them like in Figure \ref{fig:cascade_hall}. One might think that each row or column constitutes a stage, but this is not the case. Stages are determined by the piping between cascades. A cascade can be imagined, roughly, as a rhombus (see Figure \ref{fig:ideal_cascade}). Is it broader in the middle at stage 0, where uranium hexafluoride is fed into the apparatus. A cascade narrows in each direction as stages become smaller as material becomes more and more concentrated.  

\iffalse
\begin{figure}%[htbp!]
%\begin{center}
\includegraphics[scale=0.7]{./figs/ideal_cascade.png}
%\end{center}
\caption{Diagram of an Ideal Cascade by Avery and Davies \TODO{Cite in caption?}}
\label{fig:ideal_cascade}
\end{figure}
\fi

\iffalse
\begin{figure}%[htbp!]
%\begin{center}
\includegraphics[scale=0.7]{./figs/cascade_hall.png}
%\end{center}
\caption{A US Department of Energy Photo of a Gas Centrifuge Cascade \TODO{Figure Copyrights?}}
\label{fig:cascade_hall}
\end{figure}
\fi

There is a healthy body of research on the ideal cascade design and the trade-offs between types of cascades. Most of this focus is theoretical however, according to DG Avery and E Davies “An ideal cascade is not completely achievable in practice, because the minimum number of elements to achieve the separation objective is usually non-integral in most stages of the cascade. ”\cite{avery_1973} % 	 Uranium Enrichment by Gas Centrifuge. D G Avery, E Davies. 1973. London, UK.
As a result “practical cascades” are an ideal cascade which is followed as closely as can be done using integral numbers of centrifuges.

\section{Designing a Cascade}
An ideal cascade can be designed from the known performance of the composite centrifuges. First, the number of stages is determined:

\begin{equation}
  \label{eqn:n_stages}

\end{equation}


Next the steady-state flow flows into each stage are calculated by solving a system of linear equations.

\TODO{eqns for current stage, previous, next, example of linear eqn}

Finally, layout the machines in each stage:

\TODO{equation for machnes per stage}


\TODO{To make these calculations, you will need to know WasteAssayByAlpha, ProductAssayByAlpha}

\TODO{check python code to see if I missed anything}





The total number of centrifuges needed to achieve a desired total cascade feed rate $F_{cascade}$:
\begin{equation}
  \label{eqn:n_mach}
  \#_{machines} = \frac{\delta U_{cascade}}{\delta U_{machine}}
  = 
  \end{equation}


\TODO{EQN MAchines per cascade (?)}

The number of stages in a cascade is calculated using these equations for total enrichment stages (Equation 6) and stripping stages (Equation 7).

\TODO{Equation 6: Number of Enrichment, Stripping Stages}


When an optimal cascade is defined as having the fewest number of stages to achieve the desired output, then the cut, or splitting ratio is ½. This number, with information about the flow rate for feed, 

product and tails to determine the size of each stage. Each stage size must be calculated chronologically because the flow rates in stage n are dependent on the flow rates in stages n-1, n+1 and n


For more information about ideal cascades look to the chapter Cascade Theory in Topics in Applied Physics: Uranium Enrichment \cite{Cascade_?}.% Topics in Applied Physics: Uranium Enrichment. S Villani. New York. 1979.

%\input{discussion}
%\input{appendix}

%%%%%%%%%%%%%%%%%%%%%%%%%%%%%%%%%%%%%%%%%%%%%%%%%%%%%%%%%%%%%%%%%%%%%%%%%%%%%%%%
\begin{small}
\bibliographystyle{ANSurl}
\bibliography{../zotero_160516,../zotero_adds_160603,../manual_fixes,../websites_manual,../poli_sci_hoffman}  % NO SPACES BETWEEN BIB FILENAMES!
\end{small}
\end{document}
