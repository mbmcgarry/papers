%\documentclass{proc}  % 2-column format
\documentclass[12pt]{paper}
%\documentclass{ntmanuscript}
%\documentclass[review]{elsarticle}
\usepackage{mathptmx} % Nearly Times New Roman
\usepackage[acronym,toc]{glossaries}
\newacronym{MIT}{MIT}{the Massachusettes Institute of Technology}
\newacronym{UW}{UW}{University of Wisconsin}
\newacronym{UM}{UM}{University of Michigan}
\newacronym{Sandia}{Sandia}{Sandia National Laboratories}
\newacronym{US}{US}{United States}
\newacronym{HEU}{HEU}{highly enriched uranium}
\newacronym{GWe}{GWe}{gigawatt electrical}
\newacronym{LEU}{LEU}{low enriched uranium}
\newacronym{U}{U}{uranium}
\newacronym{SWU}{SWU}{separative work unit}
\newacronym{CNERG}{CNERG}{Computational Nuclear Engineering Research Group}
\newacronym{DRE}{DRE}{dynamic resource exchange}
\newacronym{UOX}{UOX}{uranium oxide}
\newacronym{MOX}{MOX}{mixed oxide}
\newacronym{SNM}{SNM}{special nuclear material}
\newacronym{WGP}{WGP}{weapons-grade plutonium}
\newacronym{NPT}{NPT}{Nuclear Nonproliferation Treaty}
\newacronym{IAEA}{IAEA}{International Atomic Energy Agency}
\newacronym{CTBT}{CTBT}{Comprehensive Nuclear-Test-Ban Treaty}
\newacronym{START}{START}{Strategic Arms Reduction Treaties}
\newacronym{FMCT}{FMCT}{Fissile Material Cutoff Treaty}
\newacronym{NWPM}{NWPM}{Nuclear Weapons Pursuit Model}
\newacronym{NWS}{NWS}{Nuclear Weapons State}
\newacronym{NNWS}{NNWS}{Non-nuclear Weapons State}
\newacronym{CVT}{CVT}{Consortium for Verification Technology}
%\newacronym{<++>}{<++>}{<++>}
%\newacronym{<++>}{<++>}{<++>}
%\newacronym{<++>}{<++>}{<++>}
%\newacronym{<++>}{<++>}{<++>}

%\makeglossaries
%%%%%%%%%%%%%%%%%%%%%%%%%%%%%%%%%%%

\usepackage{color}
\usepackage{subcaption}
\usepackage{graphicx}
\usepackage{booktabs} % nice rules for tables
\usepackage{microtype} % if using PDF
\usepackage{xspace}
\usepackage{listings}
\usepackage{textcomp}
\usepackage{multicol,tabularx,capt-of}
\usepackage{multirow}
%\usepackage{ulem}
\usepackage{pdflscape}
% Page length commands go here in the preamble
\setlength{\oddsidemargin}{-0.25in} % Left margin of 1 in + 0 in = 1 in
\setlength{\textwidth}{7in}   % Right margin of 8.5 in - 1 in - 6.5 in = 1 in
\setlength{\topmargin}{-.75in}  % Top margin of 2 in -0.75 in = 1 in
\setlength{\textheight}{9.2in}  % Lower margin of 11 in - 9 in - 1 in = 1 in



\definecolor{listinggray}{gray}{0.9}
\definecolor{lbcolor}{rgb}{0.9,0.9,0.9}
\definecolor{burgundy}{rgb}{0.5, 0.0, 0.13}
\definecolor{burntorange}{rgb}{0.8, 0.33, 0.0}
\definecolor{chromeyellow}{rgb}{1.0, 0.65, 0.0}
\definecolor{darkred}{rgb}{0.55, 0.0, 0.0}

\lstset{
    %backgroundcolor=\color{lbcolor},
    language={C++},
    tabsize=4,
    rulecolor=\color{black},
    upquote=true,
    aboveskip={1.5\baselineskip},
    belowskip={1.5\baselineskip},
    columns=fixed,
    extendedchars=true,
    breaklines=true,
    prebreak=\raisebox{0ex}[0ex][0ex]{\ensuremath{\hookleftarrow}},
    frame=single,
    showtabs=false,
    showspaces=false,
    showstringspaces=false,
    basicstyle=\scriptsize\ttfamily\color{green!40!black},
    keywordstyle=\color[rgb]{0,0,1.0},
    commentstyle=\color[rgb]{0.133,0.545,0.133},
    stringstyle=\color[rgb]{0.627,0.126,0.941},
    numberstyle=\color[rgb]{0,1,0},
    identifierstyle=\color{black},
    captionpos=t,
}

\newcommand{\code}[1]{\lstinline[basicstyle=\ttfamily\color{green!40!black}]|#1|}
\newcommand{\units}[1] {\:\text{#1}}%
\newcommand{\SN}{S$_N$}
\newcommand{\cyclus}{\textsc{Cyclus}\xspace}
\newcommand{\Cyclus}{\cyclus}
\newcommand{\citeme}{\textcolor{red}{CITE}\xspace}
\newcommand{\TODO}[1] {{\color{red}\textbf{TODO: #1}}}%

\newcommand{\comment}[1]{{\color{green}\textbf{#1}}}

%%%%%%%%%%%%%%%%%%%%%%%%%%%%%%%%%%%
\begin{document}


%\begin{frontmatter}
\title{Fundamentals of Centrifuge Enrichment: A Technical Primer}

% Authors. Separated by commas
\author{
  Meghan B. McGarry$^1$,
  Drew Buys$^1$,
  Paul P.H. Wilson$^1$}


\date{}
% Institutes of the authors
\institution{$^1$Department of Nuclear Engineering and Engineering Physics \\
University of Wisconsin - Madison}
\institution{$^2$Sandia National Laboratories}
% Information concerning the person submitting the manuscript
%\submitter{Meghan B. McGarry}
%\submitteraddress{1500 Engineering Drive, Madison, WI, USA}
%\submitteremail{mbmcgarry@wisc.edu}

% No more than three keywords, though each can be a phrase
%\keywords{fuel cycle, simulatiom, non-proliferation}
\maketitle

% I. Motivation
% x   -  Identify Factors that motivate
% x  II. Benchmark model against historical data:
% x A. Develop historical database
% x   - What sources?
% x   - What model to convert to 10pt scale?
% x        - Conflict
% x B. Determine relative weighting of factors
% x   - Calculate values for historical database
% x   - PCA to determine relative weights
% ** C. Table of State Score Results
% III. Develop forward model 
%   - From score to a likelihood, using historical data
% IV. Limitations of model
%   - small dataset
%   - threshold value for proliferation
%   - no good model for scientific network
% V. Future work
%   - Apply to case study (JCPOA?)
% VI Appendix of factor conversions

\begin{abstract}

 

\end{abstract}


%\end{frontmatter}


\section{Introduction}
\label{s_motive}

%% I have pivoted this paragraph to de-emphasize the tie between energy knowledge and bomb knowledge. Is it ok now or still to much implied?  If you don't like how it sounds feel free to edit. I am having trouble striking the right balance.
Nuclear expertise is rapidly expanding around the world as demand for energy increases steadily. Because nuclear energy is clean and carbon-neutral, climate change concerns further tilt the scales making nuclear power appealing to a growing number of countries \cite{mooney_why_2014}.  China is already investing heavily in nuclear power, planning to triple its generating capacity from 19 \gls{GWe} to 58 \gls{GWe} by 2020 \cite{_china_2014}.  As climate change becomes increasingly important with respect to national security, the perception of the risk inherent to nuclear energy is decreasing and states are embracing nuclear energy as a reliable large-scale source of carbon-neutral energy.  However, the expansion of nuclear power amplifies nuclear security concerns, because the same technologies used to produce nuclear fuel can also be exploited in the pursuit of nuclear weapons.  Moreover, in the 70 years since nuclear bombs were dropped on Hiroshima and Nagasaki, the knowledge and technology required to make these weapons has proliferated around the globe \cite{feiveson_unmaking_2014}. There are now nine states that have developed their own nuclear weapons either through indigenous research or transfer of knowledge from existing programs. As nuclear power becomes more ubiquitious, it becomes ever more important to meaningfully decouple the nuclear expertise required for the pursuit of energy from that of nuclear weapons.  

\subsection{Sensitive Parts of the Nuclear Fuel Cycle}

Two nuclear technologies are of of particular concern for proliferation, uranium enrichment and plutonium reprocessing.  Uranium enrichment is required for the once-through fuel cycles that are dominant around the world today, and used exclusively in the \gls{US}.  A once-through fuel cycle includes a source of natural uranium such as a mine, and is comprised primarily of non-fissile 99.3\% $^{238}U$, with only 0.7\% fissile $^{235}U$ that is able to undergo nuclear fission. Concentrations of 3-5\% fissile $^{235}U$ are typical for fueling a nuclear power reactor.  (Research reactors use higher levels of enrichment and there are ongoing efforts to phase out those that use enrichments above 20\%).  Enrichment facilities are used to increase the concentration of $^{235}U$ from natural stock to the desired amount.  Fuel is then burned in a nuclear reactor and the remaining material, which includes the majority of the original $^{238}U$, short- and long-lived fission products, and $\sim$1\% Pu (239 and 240), is then stored as waste.  The enrichment phase of the fuel cycle is a poliferation concern because in principle it can be used to increase the concentration of $^{235}U$ up to the 90\% or more typically used to make a nuclear weapon \cite{_military_2014}.
%% There is much debate about making weapons out of lower enrichments, probably not worth including here. -- %% I've heard a bit about using lower enrichments for bomb-making, but not much. Maybe we discuss in Ann Arbor as well.

Plutonium reprocessing is a proliferation concern because the technique can be used either to make recycled fuel or to make weapons-grade fissile material.  Several countries are developing nuclear reactors that can accomodate recycled fuel, providing the possibility of a closed fuel cycle in which the burning of nuclear fuel would at the same time generate new nuclear fuel \cite{_processing_2015}.  Recycled fuel is plutonium-based rather than uranium-based, and is made by separating the components of spent uranium fuel to extract the plutonium concentrations of fissile $^{239}Pu$. This material can then be blended with uranium to make \gls{MOX} fuel. (\gls{MOX} can also be made by sourcing the plutonium from decomissioned weapons). The concentration of $^{239}Pu$ depends on the amount of time the fuel was burned in the reactor, and can be upwards of 50\%.  Specially designed irradiation of uranium fuel can produce a plutonium component with $^{239}Pu$ concentrations up to 93\%, known as \gls{WGP}.  Reprocessing has been considered in the \gls{US} at several times over the past half-century.  However, a host of political, economic, environmental and strategic concerns have pushed the issue of reprocessing out of the technical realm and it has become a contentious political topic, currently the \gls{US} is pursuing only basic science research in this field \cite{rossin_policy_????, editorial_adieu_2009}.
%% Do you want to mention currently-under-construction US MOX facility for converting WGP to commercial MOX fuel? It may never be completed, though.
%% I think I'll leave it for question/answer session

\subsection{Use of Treaties to Curtail Proliferation}

While it has not proven possible to prevent the spread of nuclear knowledge entirely, international treaties have been used in an attempt to minimize it.  The \gls{NPT}, which has been signed by 190 states including the original five nuclear weapons states, has codified a set of rules and norms for allowing the peaceful pursuit of nuclear energy \cite{_treaty_????}.  The \gls{NPT} created the \gls{IAEA}, whose role is to verify compliance with the treaty by periodically inspecting facilities related to nuclear technology.  Other relevant treaties include \gls{CTBT}, which placed a moratorium on testing nuclear weapons, and the \gls{START} in which the \gls{US} and Russia agreed to nuclear arms reductions \cite{_treaty:_????, department_of_State_new_2010}. (The \gls{CTBT} has been signed by 164 states but has not yet entered into force).

These treaties have done much to prevent the spread of nuclear weapons knowledge, but they do not address the weapons production capabilities of states that already posess nuclear weapons.
%% These treaties are not designed to prevent the spread of nuclear knowledge, just nuclear weapons knowledge.
%% Sorry, I didn't mean to imply otherwise.
A potential \gls{FMCT} would place limits on the amount of weapons-grade fissile material that each signatory state could stockpile, possibly including current stockpiles in the case of weapons states.  However, a major unresolved issue is the difficulty of developing verification techniques to ensure compliance \cite{_fissile_2013}.  Furthermore, measuring nuclear material for treaty verification is itself a sensitive issue, as even collecting the spectra of a material to confirm its authenticity can potentially expose sensitive information to the inspecting party \cite{glaser_zero-knowledge_2014}. Particularly if non-weapon states are to contribute to treaty verification, it is important to prevent the further dissemination of nuclear weapons knowledge.

\begin{figure}%[htbp!]
\begin{center}
\includegraphics[natwidth=162bp,natheight=227bp, scale=0.45]{./figs/cyclus_interdiscipline.png}
\end{center}
\caption{The \Cyclus nuclear fuel cycle simulator provides a testbed to integrate innovations in treaty verification across many disciplines.}
\label{fig:cyclus_diagram}
\end{figure}

An effective treaty verification regime must synthesize knowledge from the realms of political science, international relations, nuclear physics and engineering, and even behavioral psychology.  Figure \ref{fig:cyclus_diagram} illustrates  the role of a fuel cycle simulator such as \Cyclus in bringing together these disparate fields to provide insights into proposed verification technologies. A fuel cycle simulator tracks the flow of nuclear material through the facilities in a fuel cycle. It creates synthetic data, such as what would be available to an inspector, for many different facilities simultaneously while incorporating a system-level perspective of proliferation scenarios. This synthetic data can then be used as a testbed to investigate the efficacy of new detection and analysis techniques. In this way, simulators can be used to  to illucidate the strengths and weaknesses of various verification strategies.



\section{Centrifuge Design}
\label{s_centrifuge}
A gas centrifuge used for uranium enrichment is much like any other centrifuge. It uses the difference in weight between two isotopes of uranium, \gls{U235} and \gls{U238} to separate out the desired isotope, \gls{U235} by spinning it at high velocity. The lighter \gls{U235} collects at the top of the centrifuge while the heavier \gls{U238} sinks to the bottom.

\iffalse
\begin{figure}%[htbp!]
%\begin{center}
\includegraphics[scale=0.7]{./figs/centrifuge.png}
%\end{center}
\caption{A Nuclear Regulatory Commission Diagram of a Simple Gas Centrifuge}
\label{fig:likely}
\end{figure}
\fi

The type of centrifuge used in uranium enrichment is almost always the counter-current gas centrifuge, a drawing of which can be seen in Figure \ref{fig:centrifuge}. Although there are other types, such as the concurrent centrifuge and evaporative centrifuge, the counter-current centrifuge is the only type prominently used in enrichment operations in Iran and Pakistan. The counter-current centrifuge is both more efficient in terms the time it takes to separate uranium and in terms of the energy it uses to do so.

A centrifuge is comprised of several constituent parts. The first thing the outside observer will notice is the large cylindrical case that houses the rest of the centrifuge. Next to draw the eye is the piping running between cases, one of which allows uranium hexafluoride to be fed into the machine as well as separate pipes for removing the product stream containing the more enriched uranium and the tails stream which contains the less enriched or depleted uranium. These pipes run through the case and into the rotor, a cylindrical tube closed on each end with cap pieces, where the separation of isotopes actually happens. Consequently, the rotor is the part which most concerns those working on non-proliferation. Its height, diameter, and the speed at which it can rotate are all important factors in calculating the \gls{SWU} of a centrifuge. 

Inside the rotor, at the very center both vertically and horizontally, is the center post where uranium enters the machine\cite{Olander_1981} %\The Theory of Uranium Centrifuge Enrichment Olander, Donald. 17 March 1981. Progress in Nuclear Energy.
At the top and bottom of the rotor are scoops which help remove separated gas from the centrifuge, depleted gas with a higher concentration of \gls{U238} from the bottom and enriched gas with a higher concentration of \gls{U235} at the top. The bottom scoop also contributes to the counter-current flow within the machine. Between the top scoop and the main chamber of the centrifuge is a device called a baffle which keeps the top scoop from interfering in the flow of the machine . Some centrifuges include a part called a bellows that connects rotor sections. This is not a bellows in the way one might imagine a medieval blacksmith using one, but a connector made of steel or carbon fiber that allows the rotor greater flexibility when being rotated at high speed.

Centrifuges must be constructed with materials that are able to withstand both the high speed at which they are rotated and the corrosive effects of UF6. These materials include aluminum alloys, maraging steel, carbon fiber and other synthetics\cite{shin_sung} % Implications of Iran's Success in Developing Fourth-Generation Centrifuge Technology Shin Sung Tack. Online Series KINU
. The typical progression in development for countries building their own centrifuges is to begin with aluminum, move to maraging steel and then to carbon fiber. There are several reasons for this. The first is that aluminum is cheaper and easier to obtain than either maraging steel or carbon fiber. This makes it better for a proof of concept centrifuge design and simpler to begin with if a state is concerned about export controls limiting their access to materials. One of the consistent barriers for Iran's nuclear program has been its inability to buy quality carbon fiber or maraging steel abroad due to international sanctions and not having the industry at home to manufacture them at a high enough quality. Aluminum is also easier to machine than either maraging steel or carbon fiber. According to the \gls{ISIS}, this has led to Iranian engineers using aluminum end caps on steel rotor tubes as a way to preserve the steel and carbon fiber they do have\cite{isis_webpage} %Iran's new centrifuge: What do we know about it? David Albright, Jacqueline Shire, Paul Brannan. April 13, 2010. Institute for Science and International Security..  

\section{Centrifuge Equations}


\begin{multline}
  \label{eqn:raetz}
  \delta U_{Raetz}(L,F,\theta,Z_{P})
  =\frac{1}{2}F\theta (1-\theta)
  \left ( \frac{\Delta M}{2RT}v_{a}^{2} \right )^{2}
  \left ( \frac{r_{2}}{a} \right )^{4}
  \left [1-\left ( \frac{r_{1}}{r_2} \right )^{2}
  \right ] ^{2} \\
* \left[
  \left( \frac{1+L/F}{\theta} \right )
  \left ( 1 - e^{-A_{p}(L,F,\theta)Z_{p}}\right )
  + \left (\frac{L/F}{1-\theta}\right )
  \left \{ 1-e^{-A_{W}(L,F,\theta)(Z - Z_{P})} \right \} 
  \right]^{2}
\end{multline}


\begin{equation}
  A_{P} = \left (
  \frac{2 \pi D_{\rho}}{ln(r_{2}/r_{1})}
  \right )
  \left (
  \frac{\theta/F}{(1+L/F)(1-\theta+L/F)}
  \right )
\end{equation}
\begin{equation}
  A_{W} = \left (
  \frac{2 \pi D_{\rho}}{ln(r_{2}/r_{1})}
  \right )
  \left (
  \frac{\theta/F}{(L/F)(1-\theta+L/F)}
  \right )
\end{equation}

\begin{equation}
  Z_{P} = \left (
  \frac{(1-\theta)(1+L/F)}{1-\theta+L/F}
  \right ) Z
\end{equation}


F = Feed flow in moles/second
P = Product flow in moles/second
$\theta$ =P/F  (cut)
$\Delta$ M=Difference in molar mass between $U_{235}F_{6}$  and  $U_{238}F_{6}$
R=universal gas constant
T=Average temperature of the gas
$v_{a}^{2}$=peripheral velocity of the rotor in meters per second
$r_{2}$=radius at which product is extracted
a=radius of rotor
$r_{1}$=radius at which waste is extracted
L=countercurrent flow rate
$D_{\rho}$ = gas diffusion constant
Z = rotor height

\begin{equation}
  \label{eqn:sep_fac}
\alpha = \sqrt{\frac{2\delta U}{\theta/(1-\theta)L}} + 1
\end{equation}

One important note for the reader: Equation \ref{eqn:raetz} is typically discussed in the literature as being in terms of molar fraction or moles of \gls{UF6} per second. Equation \ref{eqn:sep_fac}, and most of the equations in the cascade section, are in terms of atom fraction or kilograms of uranium per year. This is a discrepancy found both in the technical literature and in the non-proliferation policy world. For example, Iran often reports its \gls{SWU} capacity in terms of enriched kilograms of \gls{UF6}. In the west when Iran’s enrichment limits are in terms of kilograms of enriched uranium – because that’s the mass that counts against the limits in the \gls{JCPOA} and the 90kg necessary to build a nuclear weapon.




\section{Iran's Centrifuge Program}
\label{s_iran}

Iran currently has seven centrifuge designs under development that we know about: the IR-1, IR-2, IR-2m, IR-4, IR-5, IR-6 and IR-8. They are named IR, for Iran, and then a number, which often but doesn’t always correspond with their development order. The simplest and least effective design, the IR-1, is what Iran is able to use under the \gls{JCPOA}. This centrifuge has an effective rotor height of 1.8 meters, a diameter of 10 centimeters and is constructed primarily of aluminum. It can spin at about 335 meters per second and has a projected SWU of just under 1 per year. Iran’s most effective centrifuge design, and the most advanced centrifuge on which it is allowed to do research during the Iran deal, is the IR-8. Little is publically known about its design. A September 2014 \gls{ISIS} report suggests that it is 2.3 meters tall, spins at 550 meters per second and has a projected SWU capacity of 16 per year. Below is a table which compares information on several Iranian centrifuges with similar Pakistani centrifuges that the early Iranian designs were based on\cite{isis_website, glaser_?, tack_?,fars_?).

\begin{table}
\centering
\begin{tabular}{|c|c|c|c|c|c||c|}
\hline
Name & Rotor Height & Diameter & Velocity & Construction & Reported & Efficiency & P-type \\
     & (m)          &  (cm)    & (m/s)    &              & SWU/yr   &
\hline
IR-1  & 1.8 & 10   & 335 & Aluminum     & $\sim$0.75 & 0.49 & P1 \\
IR-2  & 0.5 & N/A  & N/A & Carbon Fiber & $\sim$5    & 0.43 & P2 \\
IR-2m & 1   & 14.5 & 550 & Carbon Fiber & 5          & N/A  & P2 \\
IR-4  & 1 & 14.5 & 550 & Carbon Fiber & $\sim$5 & N/A & P2 \\
IR-8 & 2.3 & N/A & 550 & N/A & 16 & N/A & N/A \\
P1 & 2 & 10 & N/A & Aluminum & 




\end{tabular}
\caption{A Selection of Iranian Centrifuges and their Pakistani Comparisons. N/A is not available; --- indicates not relevant}
\label{tab:factor_sources}
\end{table}


\section{Cascades}
\label{s_cascade}
A single centrifuge is limited in both the amount of uranium it can enrich and the level to which it can enrich that uranium by its separation coefficient. This limit means that an individual centrifuge cannot develop enough fuel for either a power plant or a bomb. So, in order to enrich uranium at a useful scale centrifuges are set up in cascades. A cascade is made up of groups of centrifuges split into stages that are organized by level of uranium enrichment; each stage is a set of identical centrifuges that enrich fuel from the previous stage. The cascades in each stage are connected in parallel, while the connections between stages are in series . This means that gas in centrifuges in the same stage 
A picture of a cascade would show equal sized rows and columns of centrifuges with uniform spacing between them like in Figure \ref{fig:cascade_hall}. One might think that each row or column constitutes a stage, but this is not the case. Stages are determined by the piping between cascades. A cascade can be imagined, roughly, as a rhombus (see Figure \ref{fig:ideal_cascade}). Is it broader in the middle at stage 0, where uranium hexafluoride is fed into the apparatus. A cascade narrows in each direction as stages become smaller as material becomes more and more concentrated.  

\iffalse
\begin{figure}%[htbp!]
%\begin{center}
\includegraphics[scale=0.7]{./figs/ideal_cascade.png}
%\end{center}
\caption{Diagram of an Ideal Cascade by Avery and Davies \TODO{Cite in caption?}}
\label{fig:ideal_cascade}
\end{figure}
\fi

\iffalse
\begin{figure}%[htbp!]
%\begin{center}
\includegraphics[scale=0.7]{./figs/cascade_hall.png}
%\end{center}
\caption{A US Department of Energy Photo of a Gas Centrifuge Cascade \TODO{Figure Copyrights?}}
\label{fig:cascade_hall}
\end{figure}
\fi

There is a healthy body of research on the ideal cascade design and the trade-offs between types of cascades. Most of this focus is theoretical however, according to DG Avery and E Davies “An ideal cascade is not completely achievable in practice, because the minimum number of elements to achieve the separation objective is usually non-integral in most stages of the cascade. ”\cite{avery_1973} % 	 Uranium Enrichment by Gas Centrifuge. D G Avery, E Davies. 1973. London, UK.
As a result “practical cascades” are an ideal cascade which is followed as closely as can be done using integral numbers of centrifuges.

\section{Designing a Cascade}
An ideal cascade can be designed from the known performance of the composite centrifuges. First, the number of stages is determined:

\begin{equation}
  \label{eqn:n_stages}

\end{equation}


Next the steady-state flow flows into each stage are calculated by solving a system of linear equations.

\TODO{eqns for current stage, previous, next, example of linear eqn}

Finally, layout the machines in each stage:

\TODO{equation for machnes per stage}


\TODO{To make these calculations, you will need to know WasteAssayByAlpha, ProductAssayByAlpha}

\TODO{check python code to see if I missed anything}





The total number of centrifuges needed to achieve a desired total cascade feed rate $F_{cascade}$:
\begin{equation}
  \label{eqn:n_mach}
  \#_{machines} = \frac{\delta U_{cascade}}{\delta U_{machine}}
  = 
  \end{equation}


\TODO{EQN MAchines per cascade (?)}

The number of stages in a cascade is calculated using these equations for total enrichment stages (Equation 6) and stripping stages (Equation 7).

\TODO{Equation 6: Number of Enrichment, Stripping Stages}


When an optimal cascade is defined as having the fewest number of stages to achieve the desired output, then the cut, or splitting ratio is ½. This number, with information about the flow rate for feed, 

product and tails to determine the size of each stage. Each stage size must be calculated chronologically because the flow rates in stage n are dependent on the flow rates in stages n-1, n+1 and n


For more information about ideal cascades look to the chapter Cascade Theory in Topics in Applied Physics: Uranium Enrichment \cite{Cascade_?}.% Topics in Applied Physics: Uranium Enrichment. S Villani. New York. 1979.

%\section{Discussion}
\label{s_dis}

\Cyclus is able to model signatures of diversion from a diverse set of facilities in the nuclear fuel cycle and with a variety of data modalities. Table \ref{tab:modalities} lists a variety of signal modalities and their applications in the fuel cycle (TODO: synonym for schema?).  One modality with diverse set of potential applications is satellite imagery.  We are now developing the software infrastructure to create synthetic satellite images that may contain signals of diversion. Satellite imagery has a variety of applications: tracking personnel or truck movement patterns, thermal or visible signatures of effluent or heat, or major facility changes such as new or removed buildings.

%% TODO: MAKE TABLE WITH MODALITIES AND THEIR APPLICATIONS
% Simulation parameters in RS_3sink.xml at
%/Users/mbmcgarry/git/data_analysis/data/v1.2/random_sink/
\begin{table}
\centering
\begin{tabular}{|c|c|c|}
\hline
\textbf{General}    & Duration (months)       & 100  \\
\textbf{Simulation} & Natural U (\% $^{235}U$) & 0.7  \\
\textbf{Parameters} & LEU (\% $^{235}U$)       & 4.0  \\
                    & HEU (\% $^{235}U$)       & 90.0 \\
\hline
\textbf{Enrichment} & SWU Capacity (kg-SWU/month) & 180  \\
\textbf{Facility}   & Tails Assay (\% $^{235}U$)   & 0.3  \\
\hline
\textbf{LEU Demand} & Mean Qty (kg)       & 33.0  \\
                    & $\sigma$ (kg)       & 0.5  \\
\hline
\textbf{HEU Demand} & Qty (kg)            & 0.03  \\
                    & Avg Rate of Occurrence & 1/5 \\ 
\hline
\end{tabular}
\caption{Simulation parameters for \gls{HEU} diversion scenario.}
\label{tab:sim_params}
\end{table}

These diverse datasets can be combined to highlight signatures of diversion that are small enough to be hidden in the noise of individual signals.  We have illustrated this technique by combining time-series data for power consumption and declared \gls{LEU} production for a simple scenario of \gls{HEU} production in an enrichment facility.  More realistic scenarios require advanced anomaly detection techniques such as those being developed at \gls{UM}. A collaboration with \gls{UM} and \gls{Sandia} will investigate ways to optimize subsets of diverse signal modalities to ensure reliable detection while minimizing resource usage.

The \Cyclus fuel cycle simulator is being used as a framework for combining techniques and knowledge from a variety of disciplines to support a cohesive approach to treaty verification.  Moving forward, \Cyclus will be used to study more complex and realistic diversion scenarios.  Additionally, \Cyclus has the capability to produce synthetic signals of inherent physical processes such as neutron spectra of various materials.  In this way, \Cyclus simulations can provide theoretical signals to researchers developing experimental detectors in order to test sensitivity and detector response.  \Cyclus is also being used to explore behavioral mode ***


Probabilistic models for behavior based on the actor's risk-perception will be explored.  Ongoing collaborations as part of the \gls{CVT} are examining the mechanisms and limits of expanding anomaly detection algorithms with other types of data, such as social media chatter.  Due to the inherently interdisciplinary nature of this work, new external collaborations are sought with experts in behavioral modeling. Innovative ideas on detection modalities and diversion detection techniques are also welcomed.



\textit{This work was funded in-part by the Consortium for Verification Technology under Department of Energy National Nuclear Security Administration award number DE-NA0002534”}

%\section{Appendix: Factor Conversion Tables}
\label{s_appendix}

%\iffalse

%\begin{landscape}
\begin{table}
\centering
\begin{tabular}{|c|c|c|c|c|c|c|c|c|c|}
\hline
\textbf{Factor}      & \textbf{Auth}          & \textbf{Enrich/}      & \multicolumn{2}{c|}{\textbf{Military Iso.}}          & \textbf{Mil. Spend} &  \textbf{Reactors}  & \textbf{Sci.} & \textbf{Uran.} \\
\textbf{Score}  &                        &  \textbf{Repro.}   & \multicolumn{2}{c|}{$10 - (A_{NNWS}+A_{NWS})$}  & \textbf{(\%GDP)}    &   (Power+Research) & \textbf{Net.} &  \textbf{Res} \\
\cline{4-5}
 (FS)           &                         &                       &  NNWS                 & NWS               &                    & $10 - R_{all}$&               &              \\


\hline
\textbf{0}      &  0                     &  0                   &  --                        &  --                     &  --                 &   0                   &    --         &  0           \\   
\textbf{1}      &  1                     &  --                  &  1-2                       &  --                     &  $<$ 1              &   1-3 planned         &    --         &  --          \\
\textbf{2}      &  2                     &  --                  &  3-4                       &  --                     &  [1, 2)             &   4+ planned          &    --         &  --          \\
\textbf{3}      &  3                     &  --                  &  5+                        &  --                     &  --                 &    --                 &    --         &  --          \\
\textbf{4}      &  4                     &  --                  &  --                        &  --                     &  [2, 3)             &    1-3 built          &     1         &  --          \\
\textbf{5}      &  5                     &  --                  &  --                        &  1                      &  --                 &       --              &    --         &  --          \\
\textbf{6}      &  6                     &  --                  &  --                        &  2                      &  --                 &       --              &    --         &  --          \\
\textbf{7}      &  7                     &  --                  &  --                        &  3+                     &  [3, 5)             &    4-7 built          &     2         &  --          \\
\textbf{8}      &  8                     &  --                  &  --                        &  --                     &  --                 &       --              &    --         &  --          \\
\textbf{9}      &  9                     &  --                  &  --                        &  --                     &  --                 &       --              &    --         &  --          \\
\textbf{10}     &  10                    &  1                   &  --                        &  --                     &   5.0+              &    8+ built           &     3         &  10          \\

\hline
\end{tabular}
\caption{Conversion table from raw data to final factor score (FS). Square brackets are inclusive, parentheses are exclusive, such that [1,2) indicates $1<=x<2$. Reactors and military alliances (used to define military isolation) are both anti-correlated to pursuit so the final factor score for these factors is 10 minus the value shown in the table. Conflict factor is defined separately in table \ref{tab:conflict}. }
\label{tab:factor_conversions}
\end{table}
%\end{landscape}


\begin{table}
\centering
\begin{tabular}{|c|c|c|c|}
\hline
\textbf{Nuc. Weapon Status} & \textbf{Allies}  & \textbf{Neutral}  & \textbf{Enemies} \\
\hline
NNWS - NNWS     & 2 & 2 & 6 \\
NNWS - Pursue   & 3 & 4 & 8 \\
NNWS - NWS      & 1 & 4 & 7 \\
Pursue - Pursue & 4 & 5 & 9 \\
Pursue - NWS    & 3 & 6 & 10 \\
NWS - NWS       & 1 & 3 & 5 \\
\hline
\end{tabular}
\caption{\TODO{caption}}
\label{tab:conflict}
\end{table}
%\end{landscape}


%%%%%%%%%%%%%%%%%%%%%%%%%%%%%%%%%%%%%%%%%%%%%%%%%%%%%%%%%%%%%%%%%%%%%%%%%%%%%%%%
\begin{small}
\bibliographystyle{ANSurl}
\bibliography{../zotero_160516,../zotero_adds_160603,../manual_fixes,../websites_manual,../poli_sci_hoffman}  % NO SPACES BETWEEN BIB FILENAMES!
\end{small}
\end{document}
