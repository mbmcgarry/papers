\section{Introduction}
\label{s_intro}
Under the \gls{JCPOA}, colloquially known as the Iran deal, Iran has accepted limits on its nuclear program in order to ensure the world that it is not developing nuclear weapons. Chief among these limits are restrictions on Iran’s ability to enrich uranium through its centrifuge program. Centrifuge technology is the focus of the agreement because it is the turnkey technology in developing nuclear weapons. Without a capable centrifuge enrichment program, a potential proliferant country has a much more difficult path to arm itself.

Prior to the implementation of the \gls{JCPOA}, Iran had more than 10,000 centrifuges in more than 70 cascades. The most contentious negotiations in the lead-up to agreeing to the \gls{JCPOA} surrounded the type and number of centrifuges that would be available to Iran during the agreement. It is now operating only 5,060 of its most rudimentary centrifuges (the IR-1), configured in 30 cascades, at the Natanz Fuel Enrichment Plant. The centrifuge engineering constraints in the \gls{JCPOA} limit the amount of enriched uranium Iran can produce over the course of the agreement.

The \gls{JCPOA} was designed to ensure a minimum breakout time of one year. There is no way for Iran to put more uranium in its existing centrifuges and make a bomb more quickly. To acquire a significant quantity of weapons grade uranium more quickly, Iran would either have to secretly build more centrifuges or remove them from storage facilities that are now under surveillance by the \gls{IAEA}.


\section{Nuclear Nonproliferation Basics}
To understand the policy implications of the Iran deal, it’s necessary to understand the terminology and framework in which it is discussed. Naturally occurring uranium—or natural uranium— consists of two isotopes, \gls{U235} and \gls{U238}. Natural uranium is only 0.7\% \gls{U235} by weight. A nuclear weapon requires at least 90\% \gls{U235}. Below this level of enrichment, a nuclear reaction cannot be sustained to produce an explosion.  Centrifugation is the most common way to enrich the material, increasing the concentration of \gls{U235}.

It is important to note that building a nuclear weapon is not the only reason to enrich uranium. Enriched uranium is necessary to fuel many types of civilian nuclear power reactors, and is also used in some medical and research applications. Iran claims to use its centrifuge enrichment program to create 3.5\% \gls{LEU} for nuclear power plants and 19.75\% enriched uranium, also \gls{LEU}, for medical purposes. Uranium enriched past the 20\% level, including weapons-grade uranium, is considered \gls{HEU}.

The \gls{JCPOA} is designed to hold Iran to these claims by limiting both the level to which Iran can enrich uranium and the total amount of enriched uranium allowed in the country. It takes as little as 9 kg of 90\% enriched uranium to create a sophisticated nuclear weapon. Simpler weapons require up to 50 kg\cite{ucs_2009}. %(Union of Concerned Scientists 2009)

A gas centrifuge used for uranium enrichment is much like any other centrifuge. Uranium is enriched in centrifuges in its gaseous form as \gls{UF6} , but enrichment levels and outputs are discussed in terms of uranium because that is the form in which it fuels a power plant or arms a bomb.  A centrifuge uses the difference in weight between \gls{U235} and \gls{U238} to separate out the \gls{U235} by spinning it at high velocity. The lighter \gls{U235} collects at the top of the centrifuge while the heavier \gls{U238} sinks to the bottom and is removed. A centrifuge’s ability to complete this task is its separative capacity, the energy required to enrich the uranium in the system to the desired level of enrichment. This is typically given in kilogram \gls{SWU} on uranium per unit time, which is denoted as just SWU. This measurement is the key measure of the effectiveness of a centrifuge and, consequently, was an important parameter in negotiating the \gls{JCPOA}.  

