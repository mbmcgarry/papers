\section{Iran's Centrifuge Program}
\label{s_iran}

Iran currently has seven centrifuge designs under development that we know about: the IR-1, IR-2, IR-2m, IR-4, IR-5, IR-6 and IR-8. They are named IR, for Iran, and then a number, which often but doesn’t always correspond with their development order. The simplest and least effective design, the IR-1, is what Iran is able to use under the \gls{JCPOA}. This centrifuge has an effective rotor height of 1.8 meters, a diameter of 10 centimeters and is constructed primarily of aluminum. It can spin at about 335 meters per second and has a projected SWU of just under 1 per year. Iran’s most effective centrifuge design, and the most advanced centrifuge on which it is allowed to do research during the Iran deal, is the IR-8. Little is publically known about its design. A September 2014 \gls{ISIS} report suggests that it is 2.3 meters tall, spins at 550 meters per second and has a projected SWU capacity of 16 per year. Below is a table which compares information on several Iranian centrifuges with similar Pakistani centrifuges that the early Iranian designs were based on\cite{maccalman_2016,isis_website, glaser_?, tack_?,fars_?}.%MacCalman, Molly. "A.Q. Khan Nuclear Smuggling Network." Journal of Strategic Security 9, no. 1 (2016): 104-118

\begin{table}
\centering
\begin{tabular}{|c||c|c|c|c|c|c|c|}
\hline
\textbf{Name} & \textbf{Rotor Height} & \textbf{Diameter} & \textbf{Velocity} & \textbf{Construction} & \textbf{Reported} & \textbf{Efficiency} & \textbf{P-type} \\
     & \textbf{(m)}          &  \textbf{(cm)}    & \textbf{(m/s)}    &              & \textbf{SWU/yr}   &            &        \\
\hline
\textbf{IR-1}  & 1.8 & 10   & 335 & Aluminum       & $\sim$0.75 & 0.49  & P1 \\
\hline
\textbf{IR-2}  & 0.5 & N/A  & N/A & Carbon Fiber   & $\sim$5    & 0.43  & P2 \\
\hline
\textbf{IR-2m} & 1   & 14.5 & 550 & Carbon Fiber   & 5          & N/A   & P2 \\
\hline
\textbf{IR-4}  & 1   & 14.5 & 550 & Carbon Fiber   & $\sim$5    & N/A   & P2 \\
\hline
\textbf{IR-8}  & 2.3 & N/A  & 550 & N/A            & 16         & N/A   & N/A \\
\hline
\textbf{P1}    & 2   & 10   & N/A & Aluminum       & 3          & 0.564 & --- \\
\hline
\textbf{P2}    & 1   & 15   & N/A & Maraging Steel & 6          & 0.465 & --- \\
\hline
\textbf{P3}    & 2   & 20   & N/A & Maraging Steel & 12         & N/A   & --- \\
\hline
\end{tabular}
\caption{A Selection of Iranian Centrifuges and their Pakistani Comparisons. N/A is not available; --- indicates not relevant}
\label{tab:factor_sources}
\end{table}

