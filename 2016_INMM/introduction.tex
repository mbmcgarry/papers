\section{Introduction}
\label{s_motive}


Nuclear expertise is rapidly expanding around the world as demand for energy increases steadily\cite{mooney_why_2014}.  As climate change becomes increasingly important with respect to national security, the perception of the risk inherent to nuclear energy is decreasing and states are embracing nuclear energy as a reliable large-scale source of carbon-neutral energy.  However, the expansion of nuclear power amplifies nuclear security concerns with increased production of civilian fissile materials and as the knowledge and technology required to make nuclear weapons proliferates around the globe \cite{feiveson_unmaking_2014}.  

While it has not proven possible to prevent the spread of nuclear knowledge entirely, international treaties have been used in an attempt to minimize it.  The \gls{NPT}, which has been signed by 190 states including the original five nuclear weapons states, has codified a set of rules and norms for allowing the peaceful pursuit of nuclear energy \cite{_treaty_????}.  The \gls{NPT} created the \gls{IAEA}, whose role is to verify compliance with the treaty by periodically inspecting facilities related to nuclear technology.  Other relevant treaties include \gls{CTBT}, which placed a moratorium on testing nuclear weapons, and the \gls{START} in which the \gls{US} and Russia agreed to nuclear arms reductions \cite{_treaty:_????, department_of_State_new_2010}.  A potential \gls{FMCT} would place limits on the amount of weapons-grade fissile material that each signatory state could stockpile, possibly including current stockpiles in the case of weapons states.

\begin{figure}%[htbp!]
\begin{center}
\includegraphics[natwidth=162bp,natheight=227bp, scale=0.45]{./figs/cyclus_interdiscipline.png}
\end{center}
\caption{The \Cyclus nuclear fuel cycle simulator provides a testbed to integrate innovations in treaty verification across many disciplines.}
\label{fig:cyclus_diagram}
\end{figure}

An effective treaty verification regime must synthesize knowledge from the realms of political science, international relations, nuclear physics and engineering, and even behavioral psychology.  Figure \ref{fig:cyclus_diagram} illustrates  the role of a fuel cycle simulator such as \Cyclus in bringing together these disparate fields to provide insights into proposed verification technologies. A fuel cycle simulator tracks the flow of nuclear material through the facilities in a fuel cycle\cite{huff_fundamental_2016}.  Uranium enrichment and spent fuel reprocessing are two particularly sensitive parts of the fuel cycle, but correlated signatures of illicit activity are likely to be present across the fuel cycle. A fuel cycle simulator creates synthetic data, such as what would be available to an inspector, for many different facilities simultaneously while incorporating a system-level perspective of proliferation scenarios. This synthetic data can then be used as a testbed to investigate the efficacy of new detection and analysis techniques and illucidate the strengths and weaknesses of various verification strategies.
