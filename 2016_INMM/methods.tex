\section{The \Cyclus Fuel Cycle Simulator}
\label{s_methods}

The \gls{CNERG}\footnote{http://cnerg.github.io/} group at the University of Wisconsin has developed the \Cyclus\footnote{http://fuelcycle.org/} nuclear fuel cycle simulator to model all aspects of the nuclear fuel cycle in a flexible way \cite{cyclus_v1_0,cyclus_v1_2,cyclus_v1_3}. \Cyclus produces a database file containing information on the flow of nuclear material through the fuel cycle at each timestep.  The database provides information on facility inventories, material composition, transactions between facilities, and facility build and decommissioning histories, among others.

\Cyclus is designed using an agent-based framework, meaning that each actor  in a fuel cycle (such as a mine, a nuclear reactor, or even a governing body) is modeled as an independent agent \cite{jennings_agent-based_2000, taylor2014agent}.  Each agent in the simulation is self-contained and may include physics, economics, or behavioral components \cite{huff_open_2011,gidden_agent-based_2013,gidden_agent-based_2015}.  The agents interact with one another through the \gls{DRE}, which facilitates the trading of resources and commodities \cite{gidden_agent-based_2014}.  At each timestep, agents can choose to request resources.  Resources are defined using both a quantity (e.g. 1 metric ton), and a quality, such as having a composition of 99.7\% $^{238}U$ and 0.3\% $^{235}U$.  The \gls{DRE} then solicits bids from any facilities that are interested in offering those resources. Resources can be offered as bids even if they do not exactly match the requested material. For example, a reactor might request a commodity called ``fuel'', which it has defined as being \gls{UOX}.  It may receive two bids for ``fuel'' that are specified as \gls{UOX} and \gls{MOX}, having two distinct isotopic compositions. After the bids are received, the requestor is able to state a preference for one bid over another. Finally, once the preferences have been applied, the \gls{DRE} calculates all potential trades across all agents, then executes an optimization algorithm to find the solution that most closely fulfills a maximum of requests. It is possible that as a result of the preference adjustment, no trades are executed for some facilities in a given time step.  Once this solution is found, material is transfered across the facilities and the timestep is concluded.

Although \Cyclus was designed to assess the transition from once-through fuel cycles to alternative next-generation scenarios including technologies such as spent fuel recycling, it is also an excellent tool for examining proliferation issues.  Other fuel cycle simulators designed to study energy issues are typically too inflexible to be used for other purposes. For example, they may have extremely accurate physics models of reactor burnup, but no ability to vary or modify other facilities in the fuel cycle.  \Cyclus has three key features that make it well-suited to non-proliferation studies: it is \textit{agent-based}, it tracks \textit{discrete materials}, and it incorporates \textit{social and behavioral interaction models}. Agent-based design allows for modular simulations where individual facilities can be swapped compared in otherwise identical simulations. \Cyclus tracks discrete material flow through the simulation, and uses data from PyNE,\footnote{http://pyne.io/} to track decay and transumutation data at all timesteps \cite{Scopatz2012b, huff_integrated:_2013}. Finally, behavioral modeling allows facilities and institutions to engage in dynamic decision-making based on their preferences, needs, or political constraints.  A specific agent might have preferences based on material composition, physical proximity between facilities, or allowed and disallowed trading partners, which are implemented in a region-institution-facility hierarchy that enables economic modeling \cite{oliver_geniusv2:_2009}.

Behavior is a critical aspect of non-proliferation modeling. For example, an enrichment facility receiving illicit requests for \gls{HEU} may define its own criteria for whether or not to fulfill this order.  It may disallow production of enrichments above a certain level, choose to trade only with specific facilities, or choose to preferentially fulfill requests at one enrichment level over an other.  Likewise, a requestor of \gls{HEU} can make requests at regular or random intervals, and may request randomly or gaussian distributions of material quantity.  
At the insitution level, \Cyclus allows trading decisions between facilities to be controlled by the owner of those facilities, which may be a commercial entity or a nation-state.  This facilitiates modeling of the interactions between multiple states within a region.  





