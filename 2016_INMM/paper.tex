%\documentclass{proc}  % 2-column format
\documentclass[12pt]{paper}
%\documentclass{ntmanuscript}
%\documentclass[review]{elsarticle}
\usepackage{mathptmx} % Nearly Times New Roman
\usepackage[acronym,toc]{glossaries}
\newacronym{MIT}{MIT}{the Massachusettes Institute of Technology}
\newacronym{UW}{UW}{University of Wisconsin}
\newacronym{UM}{UM}{University of Michigan}
\newacronym{Sandia}{Sandia}{Sandia National Laboratories}
\newacronym{US}{US}{United States}
\newacronym{HEU}{HEU}{highly enriched uranium}
\newacronym{GWe}{GWe}{gigawatt electrical}
\newacronym{LEU}{LEU}{low enriched uranium}
\newacronym{U}{U}{uranium}
\newacronym{SWU}{SWU}{separative work unit}
\newacronym{CNERG}{CNERG}{Computational Nuclear Engineering Research Group}
\newacronym{DRE}{DRE}{dynamic resource exchange}
\newacronym{UOX}{UOX}{uranium oxide}
\newacronym{MOX}{MOX}{mixed oxide}
\newacronym{SNM}{SNM}{special nuclear material}
\newacronym{WGP}{WGP}{weapons-grade plutonium}
\newacronym{NPT}{NPT}{Nuclear Nonproliferation Treaty}
\newacronym{IAEA}{IAEA}{International Atomic Energy Agency}
\newacronym{CTBT}{CTBT}{Comprehensive Nuclear-Test-Ban Treaty}
\newacronym{START}{START}{Strategic Arms Reduction Treaties}
\newacronym{FMCT}{FMCT}{Fissile Material Cutoff Treaty}
\newacronym{NWPM}{NWPM}{Nuclear Weapons Pursuit Model}
\newacronym{NWS}{NWS}{Nuclear Weapons State}
\newacronym{NNWS}{NNWS}{Non-nuclear Weapons State}
\newacronym{CVT}{CVT}{Consortium for Verification Technology}
%\newacronym{<++>}{<++>}{<++>}
%\newacronym{<++>}{<++>}{<++>}
%\newacronym{<++>}{<++>}{<++>}
%\newacronym{<++>}{<++>}{<++>}

%\makeglossaries
%%%%%%%%%%%%%%%%%%%%%%%%%%%%%%%%%%%

\usepackage{color}
\usepackage{subcaption}
\usepackage{graphicx}
\usepackage{booktabs} % nice rules for tables
\usepackage{microtype} % if using PDF
\usepackage{xspace}
\usepackage{listings}
\usepackage{textcomp}
%\usepackage{ulem}

% Page length commands go here in the preamble
\setlength{\oddsidemargin}{-0.25in} % Left margin of 1 in + 0 in = 1 in
\setlength{\textwidth}{7in}   % Right margin of 8.5 in - 1 in - 6.5 in = 1 in
\setlength{\topmargin}{-.75in}  % Top margin of 2 in -0.75 in = 1 in
\setlength{\textheight}{9.2in}  % Lower margin of 11 in - 9 in - 1 in = 1 in



\definecolor{listinggray}{gray}{0.9}
\definecolor{lbcolor}{rgb}{0.9,0.9,0.9}
\lstset{
    %backgroundcolor=\color{lbcolor},
    language={C++},
    tabsize=4,
    rulecolor=\color{black},
    upquote=true,
    aboveskip={1.5\baselineskip},
    belowskip={1.5\baselineskip},
    columns=fixed,
    extendedchars=true,
    breaklines=true,
    prebreak=\raisebox{0ex}[0ex][0ex]{\ensuremath{\hookleftarrow}},
    frame=single,
    showtabs=false,
    showspaces=false,
    showstringspaces=false,
    basicstyle=\scriptsize\ttfamily\color{green!40!black},
    keywordstyle=\color[rgb]{0,0,1.0},
    commentstyle=\color[rgb]{0.133,0.545,0.133},
    stringstyle=\color[rgb]{0.627,0.126,0.941},
    numberstyle=\color[rgb]{0,1,0},
    identifierstyle=\color{black},
    captionpos=t,
}

\newcommand{\code}[1]{\lstinline[basicstyle=\ttfamily\color{green!40!black}]|#1|}
\newcommand{\units}[1] {\:\text{#1}}%
\newcommand{\SN}{S$_N$}
\newcommand{\cyclus}{\textsc{Cyclus}\xspace}
\newcommand{\Cyclus}{\cyclus}
\newcommand{\citeme}{\textcolor{red}{CITE}\xspace}
\newcommand{\TODO}[1] {{\color{red}\textbf{TODO: #1}}}%

\newcommand{\comment}[1]{{\color{green}\textbf{#1}}}

%%%%%%%%%%%%%%%%%%%%%%%%%%%%%%%%%%%
\begin{document}


%\begin{frontmatter}
\title{Cyclus As a Synthetic Testbed of Systems-Level Diversion Signatures}

% Authors. Separated by commas
\author{
  Meghan B. McGarry$^1$,
  Drew Buys$^1$,
  Paul P.H. Wilson$^1$}


\date{}
% Institutes of the authors
\institution{$^1$Department of Nuclear Engineering and Engineering Physics \\
University of Wisconsin - Madison}
\institution{$^2$Sandia National Laboratories}
% Information concerning the person submitting the manuscript
%\submitter{Meghan B. McGarry}
%\submitteraddress{1500 Engineering Drive, Madison, WI, USA}
%\submitteremail{mbmcgarry@wisc.edu}

% No more than three keywords, though each can be a phrase
%\keywords{fuel cycle, simulatiom, non-proliferation}
\maketitle



\begin{abstract}

  Already the dominant source of clean energy, nuclear power is growing at a
  rapid pace.  While beneficial to a world confronting climate change, the
  nuclear security and non-proliferation impacts of expanding nuclear power
  will become more consequential.  As a result, it is imperative to develop
  credible methods to verify compliance with treaties that control fissile
  material production, such as the Non-Proliferation Treaty or a potential
  Fissile Material Cutoff Treaty. As part of the Consortium for Verification
  Technology, the Cyclus fuel cycle simulator is being used as a testbed for the
  development of new technologies and analysis approaches to treaty
  verification. Cyclus is an agent-based, systems-level simulator that tracks
  discrete material flow through the entire fuel cycle, from mining through
  burnup in reactors to a repository, or alternatively through one or more
  iterations of reprocessing. A systems-level view facilitates the study of
  correlated signals from different facilities that combine to form identifiable
  signatures of clandestine activity. Cyclus also includes a
  region/institution/facility hierarchy that can incorporate the effects of
  tariffs and sanctions in regional or global contexts.  Cyclus enables social-
  behavioral modeling of the interactions between individual facilities or
  regions.  This paper presents the first use of Cyclus to simulate nuclear
  material diversion from the fuel cycle using a variety of contemporaneous signals:
  material flow, facility power consumption, effluent emissions (including
  geospatial distribution), event-logs.  Multiple signal modalities can be
  analyzed in concert using anomaly detection techniques to identify signatures
  of material diversion or other signatures of clandestine nuclear weapons
  development.  The Cyclus testbed can then be used to examine treaty
  verification techniques and inspection regimens to to inform their sensitivity
  and limitations.

\end{abstract}


%\end{frontmatter}


\section{Introduction}
\label{s_motive}

%% I have pivoted this paragraph to de-emphasize the tie between energy knowledge and bomb knowledge. Is it ok now or still to much implied?  If you don't like how it sounds feel free to edit. I am having trouble striking the right balance.
Nuclear expertise is rapidly expanding around the world as demand for energy increases steadily. Because nuclear energy is clean and carbon-neutral, climate change concerns further tilt the scales making nuclear power appealing to a growing number of countries \cite{mooney_why_2014}.  China is already investing heavily in nuclear power, planning to triple its generating capacity from 19 \gls{GWe} to 58 \gls{GWe} by 2020 \cite{_china_2014}.  As climate change becomes increasingly important with respect to national security, the perception of the risk inherent to nuclear energy is decreasing and states are embracing nuclear energy as a reliable large-scale source of carbon-neutral energy.  However, the expansion of nuclear power amplifies nuclear security concerns, because the same technologies used to produce nuclear fuel can also be exploited in the pursuit of nuclear weapons.  Moreover, in the 70 years since nuclear bombs were dropped on Hiroshima and Nagasaki, the knowledge and technology required to make these weapons has proliferated around the globe \cite{feiveson_unmaking_2014}. There are now nine states that have developed their own nuclear weapons either through indigenous research or transfer of knowledge from existing programs. As nuclear power becomes more ubiquitious, it becomes ever more important to meaningfully decouple the nuclear expertise required for the pursuit of energy from that of nuclear weapons.  

\subsection{Sensitive Parts of the Nuclear Fuel Cycle}

Two nuclear technologies are of of particular concern for proliferation, uranium enrichment and plutonium reprocessing.  Uranium enrichment is required for the once-through fuel cycles that are dominant around the world today, and used exclusively in the \gls{US}.  A once-through fuel cycle includes a source of natural uranium such as a mine, and is comprised primarily of non-fissile 99.3\% $^{238}U$, with only 0.7\% fissile $^{235}U$ that is able to undergo nuclear fission. Concentrations of 3-5\% fissile $^{235}U$ are typical for fueling a nuclear power reactor.  (Research reactors use higher levels of enrichment and there are ongoing efforts to phase out those that use enrichments above 20\%).  Enrichment facilities are used to increase the concentration of $^{235}U$ from natural stock to the desired amount.  Fuel is then burned in a nuclear reactor and the remaining material, which includes the majority of the original $^{238}U$, short- and long-lived fission products, and $\sim$1\% Pu (239 and 240), is then stored as waste.  The enrichment phase of the fuel cycle is a poliferation concern because in principle it can be used to increase the concentration of $^{235}U$ up to the 90\% or more typically used to make a nuclear weapon \cite{_military_2014}.
%% There is much debate about making weapons out of lower enrichments, probably not worth including here. -- %% I've heard a bit about using lower enrichments for bomb-making, but not much. Maybe we discuss in Ann Arbor as well.

Plutonium reprocessing is a proliferation concern because the technique can be used either to make recycled fuel or to make weapons-grade fissile material.  Several countries are developing nuclear reactors that can accomodate recycled fuel, providing the possibility of a closed fuel cycle in which the burning of nuclear fuel would at the same time generate new nuclear fuel \cite{_processing_2015}.  Recycled fuel is plutonium-based rather than uranium-based, and is made by separating the components of spent uranium fuel to extract the plutonium concentrations of fissile $^{239}Pu$. This material can then be blended with uranium to make \gls{MOX} fuel. (\gls{MOX} can also be made by sourcing the plutonium from decomissioned weapons). The concentration of $^{239}Pu$ depends on the amount of time the fuel was burned in the reactor, and can be upwards of 50\%.  Specially designed irradiation of uranium fuel can produce a plutonium component with $^{239}Pu$ concentrations up to 93\%, known as \gls{WGP}.  Reprocessing has been considered in the \gls{US} at several times over the past half-century.  However, a host of political, economic, environmental and strategic concerns have pushed the issue of reprocessing out of the technical realm and it has become a contentious political topic, currently the \gls{US} is pursuing only basic science research in this field \cite{rossin_policy_????, editorial_adieu_2009}.
%% Do you want to mention currently-under-construction US MOX facility for converting WGP to commercial MOX fuel? It may never be completed, though.
%% I think I'll leave it for question/answer session

\subsection{Use of Treaties to Curtail Proliferation}

While it has not proven possible to prevent the spread of nuclear knowledge entirely, international treaties have been used in an attempt to minimize it.  The \gls{NPT}, which has been signed by 190 states including the original five nuclear weapons states, has codified a set of rules and norms for allowing the peaceful pursuit of nuclear energy \cite{_treaty_????}.  The \gls{NPT} created the \gls{IAEA}, whose role is to verify compliance with the treaty by periodically inspecting facilities related to nuclear technology.  Other relevant treaties include \gls{CTBT}, which placed a moratorium on testing nuclear weapons, and the \gls{START} in which the \gls{US} and Russia agreed to nuclear arms reductions \cite{_treaty:_????, department_of_State_new_2010}. (The \gls{CTBT} has been signed by 164 states but has not yet entered into force).

These treaties have done much to prevent the spread of nuclear weapons knowledge, but they do not address the weapons production capabilities of states that already posess nuclear weapons.
%% These treaties are not designed to prevent the spread of nuclear knowledge, just nuclear weapons knowledge.
%% Sorry, I didn't mean to imply otherwise.
A potential \gls{FMCT} would place limits on the amount of weapons-grade fissile material that each signatory state could stockpile, possibly including current stockpiles in the case of weapons states.  However, a major unresolved issue is the difficulty of developing verification techniques to ensure compliance \cite{_fissile_2013}.  Furthermore, measuring nuclear material for treaty verification is itself a sensitive issue, as even collecting the spectra of a material to confirm its authenticity can potentially expose sensitive information to the inspecting party \cite{glaser_zero-knowledge_2014}. Particularly if non-weapon states are to contribute to treaty verification, it is important to prevent the further dissemination of nuclear weapons knowledge.

\begin{figure}%[htbp!]
\begin{center}
\includegraphics[natwidth=162bp,natheight=227bp, scale=0.45]{./figs/cyclus_interdiscipline.png}
\end{center}
\caption{The \Cyclus nuclear fuel cycle simulator provides a testbed to integrate innovations in treaty verification across many disciplines.}
\label{fig:cyclus_diagram}
\end{figure}

An effective treaty verification regime must synthesize knowledge from the realms of political science, international relations, nuclear physics and engineering, and even behavioral psychology.  Figure \ref{fig:cyclus_diagram} illustrates  the role of a fuel cycle simulator such as \Cyclus in bringing together these disparate fields to provide insights into proposed verification technologies. A fuel cycle simulator tracks the flow of nuclear material through the facilities in a fuel cycle. It creates synthetic data, such as what would be available to an inspector, for many different facilities simultaneously while incorporating a system-level perspective of proliferation scenarios. This synthetic data can then be used as a testbed to investigate the efficacy of new detection and analysis techniques. In this way, simulators can be used to  to illucidate the strengths and weaknesses of various verification strategies.



\section{The \Cyclus Fuel Cycle Simulator}
\label{s_methods}

The \gls{CNERG}\footnote{http://cnerg.github.io/} group at the University of Wisconsin has developed the \Cyclus\footnote{http://fuelcycle.org/} nuclear fuel cycle simulator to model all aspects of the nuclear fuel cycle in a flexible way \cite{cyclus_v1_0,cyclus_v1_2,cyclus_v1_3}. \Cyclus produces a database file containing information on the flow of nuclear material through the fuel cycle at each timestep.  The database provides information on facility inventories, material composition, transactions between facilities, and facility build and decommissioning histories, among others.

\Cyclus is designed using an agent-based framework, meaning that each actor  in a fuel cycle (such as a mine, a nuclear reactor, or even a governing body) is modeled as an independent agent \cite{jennings_agent-based_2000, taylor2014agent}.  Each agent in the simulation is self-contained and may include physics, economics, or behavioral components \cite{huff_open_2011,gidden_agent-based_2013,gidden_agent-based_2015}.  The agents interact with one another through the \gls{DRE}, which facilitates the trading of resources and commodities \cite{gidden_agent-based_2014}.  At each timestep, agents can choose to request resources.  Resources are defined using both a quantity (e.g. 1 metric ton), and a quality, such as having a composition of 99.7\% $^{238}U$ and 0.3\% $^{235}U$.  The \gls{DRE} then solicits bids from any facilities that are interested in offering those resources. Resources can be offered as bids even if they do not exactly match the requested material. For example, a reactor might request a commodity called ``fuel'', which it has defined as being \gls{UOX}.  It may receive two bids for ``fuel'' that are specified as \gls{UOX} and \gls{MOX}, having two distinct isotopic compositions. After the bids are received, the requestor is able to state a preference for one bid over another. Finally, once the preferences have been applied, the \gls{DRE} calculates all potential trades across all agents, then executes an optimization algorithm to find the solution that most closely fulfills a maximum of requests. It is possible that as a result of the preference adjustment, no trades are executed for some facilities in a given time step.  Once this solution is found, material is transfered across the facilities and the timestep is concluded.

Although \Cyclus was designed to assess the transition from once-through fuel cycles to alternative next-generation scenarios including technologies such as spent fuel recycling, it is also an excellent tool for examining proliferation issues.  Other fuel cycle simulators designed to study energy issues are typically too inflexible to be used for other purposes. For example, they may have extremely accurate physics models of reactor burnup, but no ability to vary or modify other facilities in the fuel cycle.  \Cyclus has three key features that make it well-suited to non-proliferation studies: it is \textit{agent-based}, it tracks \textit{discrete materials}, and it incorporates \textit{social and behavioral interaction models}. Agent-based design allows for modular simulations where individual facilities can be swapped compared in otherwise identical simulations. \Cyclus tracks discrete material flow through the simulation, and uses data from PyNE,\footnote{http://pyne.io/} to track decay and transumutation data at all timesteps \cite{Scopatz2012b, huff_integrated:_2013}. Finally, behavioral modeling allows facilities and institutions to engage in dynamic decision-making based on their preferences, needs, or political constraints.  A specific agent might have preferences based on material composition, physical proximity between facilities, or allowed and disallowed trading partners, which are implemented in a region-institution-facility hierarchy that enables economic modeling \cite{oliver_geniusv2:_2009}.

Behavior is a critical aspect of non-proliferation modeling. For example, an enrichment facility receiving illicit requests for \gls{HEU} may define its own criteria for whether or not to fulfill this order.  It may disallow production of enrichments above a certain level, choose to trade only with specific facilities, or choose to preferentially fulfill requests at one enrichment level over an other.  Likewise, a requestor of \gls{HEU} can make requests at regular or random intervals, and may request randomly or gaussian distributions of material quantity.  
At the insitution level, \Cyclus allows trading decisions between facilities to be controlled by the owner of those facilities, which may be a commercial entity or a nation-state.  This facilitiates modeling of the interactions between multiple states within a region.  






\section{Signatures and Observables}
\label{s_signatures}


% TODO: ADD REFERENCES FOR THE TABLE INFORMATION

Figure \ref{fig:signatures} is a table-in-progress to identify potential signatures and observables of illicit activity at all major facilities in the nuclear fuel cycle.  Each potential signature is color coded to represent the accessibility of the data.  Green signatures are available through open, independent sources such as satellite imagery.  Blue signatures may not be publicly available but could be attained through official channels such as IAEA inspections.  Yellow signatures may or may not be made available but must be considered unreliable or unverifiable due to physical or political constraints, e.g. state declarations.  Not all of these signatures are currently collected, and it would be resource intensive to maintain comprehensive surveillance of all signatures.  For example, a truck that leaves a fuel fabrication plant should arrive at a reactor (or train station) with the fuel shipment, but it is infeasible to comprehensively track individual truck movement.  However, correlations in these individual signatures can be leveraged to overcome sparse datasets or resource constraints.


% TODO: REMOVE RED BOXES or add to legend, REMOVE purple boxes. Clean up further if needed? 
% Google spreadsheet: CVT_Research/Collaborations/Fuel Cycle Facility Signatures
\begin{figure}%[htbp!]
\begin{center}
\includegraphics[scale=0.8, angle=90]{./figs/signatures_table.pdf}
\end{center}
\caption{Table of potential signatures across the fuel cycle: measureable through open, independent sources such as satellite imagery (green), available through official inspections (blue), or potentially unreliable due to physical or political constraints (yellow).}
\label{fig:signatures}
\end{figure}

\Cyclus is being used to produce a variety of synthetic signals spanning a range of modalities.  We are currently developing signatures that would be available either publicly or via official inspections (green or blue).  As can be seen in Section \ref{s_results}, all facilities in \Cyclus automatically produce time-series data of material inventory.  Additionally, the enrichment facility reports it's SWU consumption as a time series that can be used as a rough proxy for facility power consumption. As seen in Figure \ref{fig:effluent}, \Cyclus can combine atmospheric transport models with facility effluent concentration ($I$) to track geographic dispersion. In this example, a simple atmospheric diffusion model of wind from the left is used to illustrate how a clandestine reprocessing facility (middle) could be hidden in close proximity to two declared facilities (top and bottom)\cite{simple_transport_model}.  In addition to regularly sampled time-series data, \Cyclus also models sparse, discrete-event data such as declared truck shipments from a facility.


% /Users/mbmcgarry/git/data_analysis/data/v1.3/pu_reprocess/
\begin{figure}%[htbp!]
\begin{center}
\includegraphics[natwidth=162bp,natheight=227bp, scale=0.4]{./figs/proper_diff_fr61.png}
\end{center}
\caption{Effluent transport with wind from left, hides a clandestine reprocessing facility x,y=(11,9) in close proximity to two declared facilities x,y=(1,3), (1,17)}
\label{fig:effluent}
\end{figure}

\Cyclus is also being used to model inspections at an enrichment facility that test for the presence of \gls{HEU}. IAEA inspections typically involve multiple swipe samples per location, with some likelihood of false-positive or false-negative results.\cite{INSPECTION_FALSE?}.  Figure \ref{fig:inspect} pairs inspections with undeclared truck shipments of \gls{HEU}. In this example, it is assumed that the likelihood of detecting \gls{HEU} in the enrichment facility increases with each shipment (because contamination is possible when \gls{HEU} is removed from cascades and bottled for shipping).  The undeclared \gls{HEU} shipments are shown as black bars, where amplitude incorporates the gross quantity \gls{HEU} that has been produced at the facility. The colored dots scale from yellow to red, indicating the fraction of positive swipes at an \gls{IAEA} inspector visit (assumiing a total of 10 swipes are taken per inspection, and an average of two inspections per year).  It is assumed that both false-positive and false-negative results are possible. 

% /Users/mbmcgarry/git/data_analysis/data/UM_data/multi_modal_v1.3/single_runs/inspect_example
\begin{figure}%[htbp!]
\begin{center}
\includegraphics[natwidth=162bp,natheight=227bp, scale=0.6]{./figs/mm_5enrich_tinytails_inspinspect_ship.png}
\end{center}
\caption{Black bars indicate clandestine shipments of \gls{HEU}, amplitudes show gross \gls{HEU} production at the enrichment facility.  Colored circles are fraction of swipes (out of 10) testing positive for \gls{HEU} during an inspection. Both false positives and false negatives may be present, introducing uncertainty into the data. As more \gls{HEU} is produced, the detection rate increases.}
\label{fig:inspect}
\end{figure}

Figure \ref{false_inspect} shows the breakdown of true positive, false positive, and false negative results for each inspection (for illustrative purposes only, in practice the true accuracy of a particular set of field data is unknowable).  Before month 30, no \gls{HEU} has been produced in the facility, so any ``positive'' swipes are false-positives (tan). Once the facility begins producing \gls{HEU}, liklihood of contamination increases until it is detected near month 80. If the swipe tests were perfect, there would be a 100\% detection rate for all inspections after that time. However, the possibility of false-negatives (pink), make it so that the measured inspection data appears to be the tan before month 80 and the red after month 80.  In this scenario, it would be difficult for an inspector to conclude that \gls{HEU} was being produced with this dataset alone.

\begin{figure}%[htbp!]
\begin{center}
\includegraphics[natwidth=162bp,natheight=227bp, scale=0.6]{./figs/mm_5enrich_tinytails_inspswipe_rates.png}
\end{center}
\caption{The scenario in Figure \ref{fig:signatures} had a 30\% rate for both false-positives and false-negatives. Before month 30, no HEU has been produced so all detections are false positives (tan). Once \gls{HEU} contamination is present (near month 80), true detections (red) combine with false-negatives (pink), resulting in a effective 70\% detection rate.}
\label{fig:false_inspect}
\end{figure}

\section{A Diversion Scenario: Highly Enriched Uranium}
\label{s_results}


% TODO: Actual #s: qty HEU, qty LEU, Power units, gaussian params etc.

% TODO: DIVERSION SCENARIO LAYOUT
\begin{figure}%[htbp!]
\begin{center}
\includegraphics[natwidth=162bp,natheight=227bp, scale=0.6]{./figs/**.png}
\end{center}
\caption{TODO:CAPTION}
\label{fig:scenario_layout}
\end{figure}


As a part of the \gls{CVT}\footnote{http://cvt.engin.umich.edu/}, \Cyclus is being used to generate multi-modal datasets with signatures of diversion with the goal of improving anomaly detection techniques. In the simplest implementation, \gsls{HEU} is clandestinely produced and then diverted from an enrichment facility.  Figure \ref{fig:scenario_layout} illustrates a toy model of this portion of the fuel cycle. A facility such as a mine supplies natural uranium (0.7\% $^{235}$U) to an enrichment facility. The enrichment facility in turn receives requests for \gls{LEU} of various enrichments from several declared light-water reactors (ignoring the fuel fabrication facility for simplicity).  The enrichment facility also receives requests for 90\% enriched \gls{HEU} from an undeclared actor seeking to build a nuclear weapon. Material production for each facility is calculated once each month for a total of 100 months. At each timestep, the enrichment facility fulfills an order for one \gls{LEU} enrichment level, and sometimes produces small quantities of \gls{HEU} request. 

% TODO: Double-plot: Power consumption and LEU production (with HEU production echoed on plot)
% ~/git/data_analysis/data/UM_data/multi_modal_v1.3/single_runs/INMM_051316

\begin{figure}%[htbp!]
\begin{center}
\includegraphics[natwidth=162bp,natheight=227bp, scale=0.6]{./figs/**.png}
\end{center}
\caption{TODO:CAPTION}
\label{fig:time_series}
\end{figure}

Figure \ref{fig:time_series} shows the time series data for declared production of \gls{LEU} (top) and the total SWU consumed by the enrichment plant (bottom) available to the inspector (SWU can be used as a rough proxy for power consumption).  Months where \gls{HEU} is produced are denoted with green on the \gls{LEU} plot.  The \gls{HEU} signature is hidden in the material flow data because there is a gaussian variation (TODO: PARAMS) in the tails assay.  Therefore it is not possible to detect diversion from the individual time-series data alone. However, the two signals can be combined to highlight the correlated signature of diverion. Figure \ref{fig:power_qty} shows time-series data for the ratio of power-consumption to declared \gls{LEU} production.  When the variation due to tails assay (``noise'') is sufficiently small, the deviations can be seen by eye (TODO: FOR EXAMPLE, AT T= ???).

% TODO: Ratio of LEU/POWER showing clear signature of diversion
\begin{figure}%[htbp!]
\begin{center}
\includegraphics[natwidth=162bp,natheight=227bp, scale=0.6]{./figs/**.png}
\end{center}
\caption{TODO:CAPTION}
\label{fig:time_series}
\end{figure}

While this example is clearly a toy model, it illustrates the power of combining multiple signals from a scenario to improve detection capabilities. In practice, as the noise increases or the \gls{HEU} quantity reduces in amplitude, this signature quickly becomes difficult to detect by eye. An important application of \Cyclus is to produce more complex synthetic datasets which are then provided to groups that specialize in developing advanced anomaly detection techniques. An ongoing collaboration with researchers at the Michigan Institute for Data Science\footnote{http://midas.umich.edu/} seeks to apply innovative anomaly detection techniques to these simulations to investigate detection limits for scenarios with sparse data sets or low signal-to-noise ratios.


\section{Discussion}
\label{s_dis}

\Cyclus is able to model signatures of diversion from a diverse set of facilities in the nuclear fuel cycle and with a variety of data modalities. Table \ref{tab:modalities} lists a variety of signal modalities and their applications in the fuel cycle (TODO: synonym for schema?).  One modality with diverse set of potential applications is satellite imagery.  We are now developing the software infrastructure to create synthetic satellite images that may contain signals of diversion. Satellite imagery has a variety of applications: tracking personnel or truck movement patterns, thermal or visible signatures of effluent or heat, or major facility changes such as new or removed buildings.

%% TODO: MAKE TABLE WITH MODALITIES AND THEIR APPLICATIONS
% Simulation parameters in RS_3sink.xml at
%/Users/mbmcgarry/git/data_analysis/data/v1.2/random_sink/
\begin{table}
\centering
\begin{tabular}{|c|c|c|}
\hline
\textbf{General}    & Duration (months)       & 100  \\
\textbf{Simulation} & Natural U (\% $^{235}U$) & 0.7  \\
\textbf{Parameters} & LEU (\% $^{235}U$)       & 4.0  \\
                    & HEU (\% $^{235}U$)       & 90.0 \\
\hline
\textbf{Enrichment} & SWU Capacity (kg-SWU/month) & 180  \\
\textbf{Facility}   & Tails Assay (\% $^{235}U$)   & 0.3  \\
\hline
\textbf{LEU Demand} & Mean Qty (kg)       & 33.0  \\
                    & $\sigma$ (kg)       & 0.5  \\
\hline
\textbf{HEU Demand} & Qty (kg)            & 0.03  \\
                    & Avg Rate of Occurrence & 1/5 \\ 
\hline
\end{tabular}
\caption{Simulation parameters for \gls{HEU} diversion scenario.}
\label{tab:sim_params}
\end{table}

These diverse datasets can be combined to highlight signatures of diversion that are small enough to be hidden in the noise of individual signals.  We have illustrated this technique by combining time-series data for power consumption and declared \gls{LEU} production for a simple scenario of \gls{HEU} production in an enrichment facility.  More realistic scenarios require advanced anomaly detection techniques such as those being developed at \gls{UM}. A collaboration with \gls{UM} and \gls{Sandia} will investigate ways to optimize subsets of diverse signal modalities to ensure reliable detection while minimizing resource usage.

The \Cyclus fuel cycle simulator is being used as a framework for combining techniques and knowledge from a variety of disciplines to support a cohesive approach to treaty verification.  Moving forward, \Cyclus will be used to study more complex and realistic diversion scenarios.  Additionally, \Cyclus has the capability to produce synthetic signals of inherent physical processes such as neutron spectra of various materials.  In this way, \Cyclus simulations can provide theoretical signals to researchers developing experimental detectors in order to test sensitivity and detector response.  \Cyclus is also being used to explore behavioral mode ***


Probabilistic models for behavior based on the actor's risk-perception will be explored.  Ongoing collaborations as part of the \gls{CVT} are examining the mechanisms and limits of expanding anomaly detection algorithms with other types of data, such as social media chatter.  Due to the inherently interdisciplinary nature of this work, new external collaborations are sought with experts in behavioral modeling. Innovative ideas on detection modalities and diversion detection techniques are also welcomed.



\textit{This work was funded in-part by the Consortium for Verification Technology under Department of Energy National Nuclear Security Administration award number DE-NA0002534”}


%%%%%%%%%%%%%%%%%%%%%%%%%%%%%%%%%%%%%%%%%%%%%%%%%%%%%%%%%%%%%%%%%%%%%%%%%%%%%%%%
\begin{small}
\bibliographystyle{ANSurl}
\bibliography{../zotero_160516,../zotero_adds_160603,../manual_fixes,../websites_manual}
\end{small}
\end{document}
