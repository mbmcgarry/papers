%\documentclass{proc}  % 2-column format
\documentclass[12pt]{paper}
%\documentclass{ntmanuscript}
%\documentclass[review]{elsarticle}
\usepackage[acronym,toc]{glossaries}
\newacronym{UW}{UW}{University of Wisconsin}
\newacronym{UM}{UM}{University of Michigan}
\newacronym{US}{US}{United States}
\newacronym{HEU}{HEU}{highly enriched uranium}
\newacronym{LEU}{LEU}{low enriched uranium}
\newacronym{U}{U}{uranium}
\newacronym{U235}{U-235}{uranium 235}
\newacronym{U238}{U-238}{uranium 238}
\newacronym{UF6}{UF6}{uranium hexafluoride}
\newacronym{SWU}{SWU}{separative work unit}
\newacronym{CNERG}{CNERG}{Computational Nuclear Engineering Research Group}
\newacronym{WGP}{WGP}{weapons-grade plutonium}
\newacronym{NPT}{NPT}{Nuclear Nonproliferation Treaty}
\newacronym{IAEA}{IAEA}{International Atomic Energy Agency}
\newacronym{JCPOA}{JCPOA}{Joint Comprehensive Plan of Action}
\newacronym{ISIS}{ISIS}{Institute for Science and International Security}
\newacronym{CVT}{CVT}{Consortium for Verification Technology}
%\newacronym{<++>}{<++>}{<++>}
%\newacronym{<++>}{<++>}{<++>}
%\newacronym{<++>}{<++>}{<++>}
%\newacronym{<++>}{<++>}{<++>}

%\makeglossaries
%%%%%%%%%%%%%%%%%%%%%%%%%%%%%%%%%%%

\usepackage{color}
\usepackage{graphicx}
\usepackage{booktabs} % nice rules for tables
\usepackage{microtype} % if using PDF
\usepackage{xspace}
\usepackage{listings}
\usepackage{textcomp}
\usepackage{ulem}

% Page length commands go here in the preamble
\setlength{\oddsidemargin}{-0.25in} % Left margin of 1 in + 0 in = 1 in
\setlength{\textwidth}{7in}   % Right margin of 8.5 in - 1 in - 6.5 in = 1 in
\setlength{\topmargin}{-.75in}  % Top margin of 2 in -0.75 in = 1 in
\setlength{\textheight}{9.2in}  % Lower margin of 11 in - 9 in - 1 in = 1 in


\definecolor{listinggray}{gray}{0.9}
\definecolor{lbcolor}{rgb}{0.9,0.9,0.9}
\lstset{
    %backgroundcolor=\color{lbcolor},
    language={C++},
    tabsize=4,
    rulecolor=\color{black},
    upquote=true,
    aboveskip={1.5\baselineskip},
    belowskip={1.5\baselineskip},
    columns=fixed,
    extendedchars=true,
    breaklines=true,
    prebreak=\raisebox{0ex}[0ex][0ex]{\ensuremath{\hookleftarrow}},
    frame=single,
    showtabs=false,
    showspaces=false,
    showstringspaces=false,
    basicstyle=\scriptsize\ttfamily\color{green!40!black},
    keywordstyle=\color[rgb]{0,0,1.0},
    commentstyle=\color[rgb]{0.133,0.545,0.133},
    stringstyle=\color[rgb]{0.627,0.126,0.941},
    numberstyle=\color[rgb]{0,1,0},
    identifierstyle=\color{black},
    captionpos=t,
}

\newcommand{\code}[1]{\lstinline[basicstyle=\ttfamily\color{green!40!black}]|#1|}
\newcommand{\units}[1] {\:\text{#1}}%
\newcommand{\SN}{S$_N$}
\newcommand{\cyclus}{\textsc{Cyclus}\xspace}
\newcommand{\Cyclus}{\cyclus}
\newcommand{\citeme}{\textcolor{red}{CITE}\xspace}
\newcommand{\TODO}[1] {{\color{red}\textbf{TODO: #1}}}%

\newcommand{\comment}[1]{{\color{green}\textbf{#1}}}

%%%%%%%%%%%%%%%%%%%%%%%%%%%%%%%%%%%
\begin{document}


%\begin{frontmatter}
\title{Modeling Material Diversion with the Cyclus Nuclear Fuel Cycle Simulator
}

% Authors. Separated by commas
\author{
  Meghan B. McGarry$^1$,
  Drew Buys$^1$,
  Paul P.H. Wilson$^1$}


\date{}
% Institutes of the authors
%\institution{$^1$Department of Nuclear Engineering and Engineering Physics,
University of Wisconsin-Madison}
%\\$^2$Sandia National Laboratories}
% Information concerning the person submitting the manuscript
%\submitter{Meghan B. McGarry}
%\submitteraddress{1500 Engineering Drive, Madison, WI, USA}
%\submitteremail{mbmcgarry@wisc.edu}

% No more than three keywords, though each can be a phrase
%\keywords{fuel cycle, simulatiom, non-proliferation}
\maketitle



\begin{abstract}

  Already the dominant source of clean energy, nuclear power is growing at a
  rapid pace.  While beneficial to a world confronting climate change, the
  nuclear security and non-proliferation impacts of expanding nuclear power
  are increasingly consequential.  As a result, it is imperative to develop
  credible methods to verify compliance with treaties that control fissile
  material production, such as the \gls{NPT} or a potential
  \gls{FMCT}. As part of the Consortium for Verification
  Technology, the Cyclus fuel cycle simulator is being used to model current
  and next-generation nuclear fuel cycles and inform treaty verification.  Cyclus
  is an agent-based, systems-level simulator that tracks discrete material flow
  through the entire fuel cycle, from mining through burnup in reactors to a
  repository, or alternatively through one or more iterations of reprocessing.
  Cyclus includes social-behavior models of individual actors, facilitating the
  study of clandestine material diversion from declared fuel cycles.  Cyclus
  also features a region-institution-facility hierarchy that can incorporate the
  effects of tariffs and sanctions in a regional or global context.  This paper
  presents initial Cyclus simulations of highly enriched uranium diversion from
  a declared once-through fuel cycle.  Material flow signals are analyzed using
  anomaly detection techniques to identify diversion.

\end{abstract}


%\end{frontmatter}


\section{Introduction}
\label{s_intro}
Under the \gls{JCPOA}, colloquially known as the Iran deal, Iran has accepted limits on its nuclear program in order to ensure the world that it is not developing nuclear weapons. Chief among these limits are restrictions on Iran’s ability to enrich uranium through its centrifuge program. Centrifuge technology is the focus of the agreement because it is the turnkey technology in developing nuclear weapons. Without a capable centrifuge enrichment program, a potential proliferant country has a much more difficult path to arm itself.

Prior to the implementation of the \gls{JCPOA}, Iran had more than 10,000 centrifuges in more than 70 cascades. The most contentious negotiations in the lead-up to agreeing to the \gls{JCPOA} surrounded the type and number of centrifuges that would be available to Iran during the agreement. It is now operating only 5,060 of its most rudimentary centrifuges (the IR-1), configured in 30 cascades, at the Natanz Fuel Enrichment Plant. The centrifuge engineering constraints in the \gls{JCPOA} limit the amount of enriched uranium Iran can produce over the course of the agreement.

The \gls{JCPOA} was designed to ensure a minimum breakout time of one year. There is no way for Iran to put more uranium in its existing centrifuges and make a bomb more quickly. To acquire a significant quantity of weapons grade uranium more quickly, Iran would either have to secretly build more centrifuges or remove them from storage facilities that are now under surveillance by the \gls{IAEA}.


\section{Nuclear Nonproliferation Basics}
To understand the policy implications of the Iran deal, it’s necessary to understand the terminology and framework in which it is discussed. Naturally occurring uranium—or natural uranium— consists of two isotopes, \gls{U235} and \gls{U238}. Natural uranium is only 0.7\% \gls{U235} by weight. A nuclear weapon requires at least 90\% \gls{U235}. Below this level of enrichment, a nuclear reaction cannot be sustained to produce an explosion.  Centrifugation is the most common way to enrich the material, increasing the concentration of \gls{U235}.

It is important to note that building a nuclear weapon is not the only reason to enrich uranium. Enriched uranium is necessary to fuel many types of civilian nuclear power reactors, and is also used in some medical and research applications. Iran claims to use its centrifuge enrichment program to create 3.5\% \gls{LEU} for nuclear power plants and 19.75\% enriched uranium, also \gls{LEU}, for medical purposes. Uranium enriched past the 20\% level, including weapons-grade uranium, is considered \gls{HEU}.

The \gls{JCPOA} is designed to hold Iran to these claims by limiting both the level to which Iran can enrich uranium and the total amount of enriched uranium allowed in the country. It takes as little as 9 kg of 90\% enriched uranium to create a sophisticated nuclear weapon. Simpler weapons require up to 50 kg\cite{ucs_2009}. %(Union of Concerned Scientists 2009)

A gas centrifuge used for uranium enrichment is much like any other centrifuge. Uranium is enriched in centrifuges in its gaseous form as \gls{UF6} , but enrichment levels and outputs are discussed in terms of uranium because that is the form in which it fuels a power plant or arms a bomb.  A centrifuge uses the difference in weight between \gls{U235} and \gls{U238} to separate out the \gls{U235} by spinning it at high velocity. The lighter \gls{U235} collects at the top of the centrifuge while the heavier \gls{U238} sinks to the bottom and is removed. A centrifuge’s ability to complete this task is its separative capacity, the energy required to enrich the uranium in the system to the desired level of enrichment. This is typically given in kilogram \gls{SWU} on uranium per unit time, which is denoted as just SWU. This measurement is the key measure of the effectiveness of a centrifuge and, consequently, was an important parameter in negotiating the \gls{JCPOA}.  


\section{A \Cyclus Model of Nuclear Weapon Pursuit}
\label{s_methods}

We have applied the factors correlated to pursuit of nuclear weapons in a regional model of state interactions using \Cyclus.  The \gls{CNERG}\footnote{http://cnerg.github.io/} group at the University of Wisconsin has developed the \Cyclus\footnote{http://fuelcycle.org/} nuclear fuel cycle simulator to model all aspects of the nuclear fuel cycle in a flexible way.  \Cyclus has three key features: it is \textit{agent-based}, it tracks \textit{discrete materials}, and it incorporates \textit{social and behavioral interaction models}\cite{jennings_agent-based_2000, gidden_agent-based_2013, taylor2014agent}. This design allows customized facilities and institutions to engage in dynamic decision-making based on their preferences, needs, or political constraints across a wide range of scenarios.A region-institution-facility framework captures have preferences based on material composition, physical proximity between facilities, or preferred trading partners.

\subsection{The Forward Model}

The region-institution-facility design has been used to develop the \gls{NWPM}. This forward model features two custom archetypes, an interaction region and a state institution\cite{mbmore}.  The state institution represents a nation-state and includes time-dynamic information about each of the pursuit factors. The interaction region is an omniscient presence in the simulation that tracks weapon status as well as interactive pursuit factors such as conflict (as described in section \ref{s_pe}), and communicates that factor to each individual state. 

Each of the motivating factors is defined for every state in a time dynamic way. Individual factors must have values between 0-10. There are several time dynamic parameterizations currently available in the model: constant, linear growth or decline, step-function (at either a specified or a randomly chosen time), or power-law.  These functions enable modeling of characteristics such as growth in military spending, development of new technologies (such as enrichment), and sudden changes to factors such as governing structure or inter-state conflict.  At each timestep, the state institution combines all of these factor values into a pursuit score using the weighted linear equation defined in section \ref{s_factors}.

 \subsection{Likelihood}

\begin{figure}%[htbp!]
%\begin{center}
\includegraphics[scale=0.7]{./figs/pe_likely.png}
%\end{center}
\caption{Fraction of states with any nuclear technology (dataset from section \ref{s_factors}) that pursued weapons at some point in the last 75 years, given their pursuit scores. Although a power-law curve (black) and a weighted linear fit (red) are equivalent with the existing data, a complete set of nation-states would increase the relative weight of lower scores, making the exponential curve a better fit.}
\label{fig:likely}
\end{figure}
 
The pursuit score ($s$) is then converted into a likelihood that the state will pursue a weapon. The relationship between pursuit score and likelihood of pursuing a weapon has been characterized using historical data of the 43 states that have developed nuclear technology since 1942. Figure \ref{fig:likely} shows the fraction of states with a given pursuit score that pursued a weapon at some point between 1942 and 2015. A power-law (red) or a linear fit (black) are  equivalently valid with the 43 state dataset. However, a global dataset of nation-states will have disproportionately lower scores and zero additional incidences of pursuit, so we consider the power-law paramaterization to be more representative.

While the historical data provides the probability of pursuit integrated over 75 years ($T$), a single-year probability is needed for the forward model. A state can only choose to pursue in a given year ($p$) if it is not already pursuing, therefore probability must be defined by the absence of pursuit. The probability that a state will Not pursue in a single years is $1 - p$. The likelihood of non-pursuit $\bar{L}$ integrated over $N$ years is:
\begin{equation}
\bar{L} = (1- p)^{N}
\end{equation}
And the likelihood of a pursuit ($L$) integrated over $N$ years becomes:
\begin{equation}
L = 1 - (1-p)^{N}
  \end{equation}
Therefore the probability of pursuit in a single year is:
\begin{equation}
p = 1 - (1 - L)^{1.0/N}
\label{eqn:likely_eqn}
\end{equation}

In the \gls{NWPM} forward model, each state's score is converted to a single year probability of pursuit using the power-law fit shown in Figure \ref{fig:likely}. The actual conversion is bounded using two Heavyside functions such that scores below a lower threshold (e.g. 4) are forced to a likelihood of zero, while scores above an upper threshold (e.g. 9) are forced to the value of the score at that threshold. At each time in the simulation, a random number generator is queried using the score-derived probability ($p$) to determine whether or not the pursuit event occurs. If a model is provided for the likelihood of acquisition given pursuit, then acquistion is also tracked. Both pursuit and acquisition decisions then influence future conflict relationships with the other states.

%If the determination is for pursuit to occur, the state deploys a secret enrichment facility. If a state is already pursuing a weapon, then the model determines whether it succeeds in acquiring a weapon at that timestep using a median time to acquire of 7.5 years, based on historical data for states that have acquired weapons. On the timestep in which a state succeeds in acquiring a weapon, highly enriched uranium is shipped out of the secret enrichment facility.


\input{results}
\input{discussion}

%%%%%%%%%%%%%%%%%%%%%%%%%%%%%%%%%%%%%%%%%%%%%%%%%%%%%%%%%%%%%%%%%%%%%%%%%%%%%%%%
\bibliographystyle{ANSurl}
\bibliography{../zotero_150923,../websites_manual}
%\bibliography{/Users/mbmcgarry/git/papers/zotero_150918}
\end{document}
