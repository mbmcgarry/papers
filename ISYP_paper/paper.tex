\documentclass{ntmanuscript}
\usepackage[acronym,toc]{glossaries}
\newacronym{MIT}{MIT}{the Massachusettes Institute of Technology}
\newacronym{UW}{UW}{University of Wisconsin}
\newacronym{UM}{UM}{University of Michigan}
\newacronym{Sandia}{Sandia}{Sandia National Laboratories}
\newacronym{US}{US}{United States}
\newacronym{HEU}{HEU}{highly enriched uranium}
\newacronym{GWe}{GWe}{gigawatt electrical}
\newacronym{LEU}{LEU}{low enriched uranium}
\newacronym{U}{U}{uranium}
\newacronym{SWU}{SWU}{separative work unit}
\newacronym{CNERG}{CNERG}{Computational Nuclear Engineering Research Group}
\newacronym{DRE}{DRE}{dynamic resource exchange}
\newacronym{UOX}{UOX}{uranium oxide}
\newacronym{MOX}{MOX}{mixed oxide}
\newacronym{SNM}{SNM}{special nuclear material}
\newacronym{WGP}{WGP}{weapons-grade plutonium}
\newacronym{NPT}{NPT}{Nuclear Nonproliferation Treaty}
\newacronym{IAEA}{IAEA}{International Atomic Energy Agency}
\newacronym{CTBT}{CTBT}{Comprehensive Nuclear-Test-Ban Treaty}
\newacronym{START}{START}{Strategic Arms Reduction Treaties}
\newacronym{FMCT}{FMCT}{Fissile Material Cutoff Treaty}
\newacronym{NWPM}{NWPM}{Nuclear Weapons Pursuit Model}
\newacronym{NWS}{NWS}{Nuclear Weapons State}
\newacronym{NNWS}{NNWS}{Non-nuclear Weapons State}
\newacronym{CVT}{CVT}{Consortium for Verification Technology}
%\newacronym{<++>}{<++>}{<++>}
%\newacronym{<++>}{<++>}{<++>}
%\newacronym{<++>}{<++>}{<++>}
%\newacronym{<++>}{<++>}{<++>}

\makeglossaries
%%%%%%%%%%%%%%%%%%%%%%%%%%%%%%%%%%%
\title{Modeling Material Diversion with the Cyclus Nuclear Fuel Cycle Simulator
}

% Authors. Separated by commas
\author{
  Meghan B. McGarry$^1$,
  Drew Buys$^1$,
  Paul P.H. Wilson$^1$}



% Institutes of the authors
\institution{$^1$Department of Nuclear Engineering and Engineering Physics \\
University of Wisconsin - Madison}
\institution{$^2$Sandia National Laboratories}

% Information concerning the person submitting the manuscript
\submitter{Meghan B. McGarry}
\submitteraddress{1500 Engineering Drive, Madison, WI, USA}
\submitteremail{mbmcgarry@wisc.edu}


% No more than three keywords, though each can be a phrase
\keywords{fuel cycle, simulation, non-proliferation}

\usepackage{color}
\usepackage{graphicx}
\usepackage{booktabs} % nice rules for tables
\usepackage{microtype} % if using PDF
\usepackage{xspace}
\usepackage{listings}
\usepackage{textcomp}
\usepackage{ulem}

\definecolor{listinggray}{gray}{0.9}
\definecolor{lbcolor}{rgb}{0.9,0.9,0.9}
\lstset{
    %backgroundcolor=\color{lbcolor},
    language={C++},
    tabsize=4,
    rulecolor=\color{black},
    upquote=true,
    aboveskip={1.5\baselineskip},
    belowskip={1.5\baselineskip},
    columns=fixed,
    extendedchars=true,
    breaklines=true,
    prebreak=\raisebox{0ex}[0ex][0ex]{\ensuremath{\hookleftarrow}},
    frame=single,
    showtabs=false,
    showspaces=false,
    showstringspaces=false,
    basicstyle=\scriptsize\ttfamily\color{green!40!black},
    keywordstyle=\color[rgb]{0,0,1.0},
    commentstyle=\color[rgb]{0.133,0.545,0.133},
    stringstyle=\color[rgb]{0.627,0.126,0.941},
    numberstyle=\color[rgb]{0,1,0},
    identifierstyle=\color{black},
    captionpos=t,
}

\newcommand{\code}[1]{\lstinline[basicstyle=\ttfamily\color{green!40!black}]|#1|}
\newcommand{\units}[1] {\:\text{#1}}%
\newcommand{\SN}{S$_N$}
\newcommand{\cyclus}{\textsc{Cyclus}\xspace}
\newcommand{\Cyclus}{\cyclus}
\newcommand{\citeme}{\textcolor{red}{CITE}\xspace}
\newcommand{\TODO}[1] {{\color{red}\textbf{TODO: #1}}}%

\newcommand{\comment}[1]{{\color{green}\textbf{#1}}}


\date{}
%%%%%%%%%%%%%%%%%%%%%%%%%%%%%%%%%%%
\begin{document}

\begin{abstract}

  Already the dominant source of clean energy, nuclear power is growing at a
  rapid pace.  While beneficial to a world confronting climate change, the
  nuclear security and non-proliferation impacts of expanding nuclear power
  will become more consequential.  As a result, it is imperative to develop
  credible methods to verify compliance with treaties that control fissile
  material production, such as the Non-Proliferation Treaty or a potential
  Fissile Material Cutoff Treaty. As part of the Consortium for Verification
  Technology, the Cyclus fuel cycle simulator is being used to model current
  and next-generation nuclear fuel cycles to inform treaty verification.  Cyclus
  is an agent-based, systems-level simulator that tracks discrete material flow
  through the entire fuel cycle, from mining through burnup in reactors to a
  repository, or alternatively through one or more iterations of reprocessing.
  Cyclus includes social-behavior models of individual actors, facilitating the
  study of clandestine material diversion from declared fuel cycles.  Cyclus
  also features a region/institution/facility hierarchy that can incorporate the
  effects of tariffs and sanctions in regional or global contexts.  This paper
  presents initial Cyclus simulations of highly enriched uranium diversion from
  a declared once-through fuel cycle.  Material flow signals are analyzed using
  anomaly detection techniques to identify diversion.

\end{abstract}

\section{Motivation}
\label{s_motive}

\section{The \Cyclus Fuel Cycle Simulator}
\label{s_methods}

The \gls{CNERG}\footnote{http://cnerg.github.io/} group at the University of Wisconsin has developed the \Cyclus\footnote{http://fuelcycle.org/} nuclear fuel cycle simulator to model all aspects of the nuclear fuel cycle in a flexible way \cite{cyclus_v1_0,cyclus_v1_2,cyclus_v1_3}. \Cyclus produces a database file containing information on the flow of nuclear material through the fuel cycle at each timestep.  The database provides information on facility inventories, material composition, transactions between facilities, and facility build and decommissioning histories, among others.

\Cyclus is designed using an agent-based framework, meaning that each actor  in a fuel cycle (such as a mine, a nuclear reactor, or even a governing body) is modeled as an independent agent \cite{jennings_agent-based_2000, taylor2014agent}.  Each agent in the simulation is self-contained and may include physics, economics, or behavioral components \cite{huff_open_2011,gidden_agent-based_2013,gidden_agent-based_2015}.  The agents interact with one another through the \gls{DRE}, which facilitates the trading of resources and commodities \cite{gidden_agent-based_2014}.  At each timestep, agents can choose to request resources.  Resources are defined using both a quantity (e.g. 1 metric ton), and a quality, such as having a composition of 99.7\% $^{238}U$ and 0.3\% $^{235}U$.  The \gls{DRE} then solicits bids from any facilities that are interested in offering those resources. Resources can be offered as bids even if they do not exactly match the requested material. For example, a reactor might request a commodity called ``fuel'', which it has defined as being \gls{UOX}.  It may receive two bids for ``fuel'' that are specified as \gls{UOX} and \gls{MOX}, having two distinct isotopic compositions. After the bids are received, the requestor is able to state a preference for one bid over another. Finally, once the preferences have been applied, the \gls{DRE} calculates all potential trades across all agents, then executes an optimization algorithm to find the solution that most closely fulfills a maximum of requests. It is possible that as a result of the preference adjustment, no trades are executed for some facilities in a given time step.  Once this solution is found, material is transfered across the facilities and the timestep is concluded.

Although \Cyclus was designed to assess the transition from once-through fuel cycles to alternative next-generation scenarios including technologies such as spent fuel recycling, it is also an excellent tool for examining proliferation issues.  Other fuel cycle simulators designed to study energy issues are typically too inflexible to be used for other purposes. For example, they may have extremely accurate physics models of reactor burnup, but no ability to vary or modify other facilities in the fuel cycle.  \Cyclus has three key features that make it well-suited to non-proliferation studies: it is \textit{agent-based}, it tracks \textit{discrete materials}, and it incorporates \textit{social and behavioral interaction models}. Agent-based design allows for modular simulations where individual facilities can be swapped compared in otherwise identical simulations. \Cyclus tracks discrete material flow through the simulation, and uses data from PyNE,\footnote{http://pyne.io/} to track decay and transumutation data at all timesteps \cite{Scopatz2012b, huff_integrated:_2013}. Finally, behavioral modeling allows facilities and institutions to engage in dynamic decision-making based on their preferences, needs, or political constraints.  A specific agent might have preferences based on material composition, physical proximity between facilities, or allowed and disallowed trading partners, which are implemented in a region-institution-facility hierarchy that enables economic modeling \cite{oliver_geniusv2:_2009}.

Behavior is a critical aspect of non-proliferation modeling. For example, an enrichment facility receiving illicit requests for \gls{HEU} may define its own criteria for whether or not to fulfill this order.  It may disallow production of enrichments above a certain level, choose to trade only with specific facilities, or choose to preferentially fulfill requests at one enrichment level over an other.  Likewise, a requestor of \gls{HEU} can make requests at regular or random intervals, and may request randomly or gaussian distributions of material quantity.  
At the insitution level, \Cyclus allows trading decisions between facilities to be controlled by the owner of those facilities, which may be a commercial entity or a nation-state.  This facilitiates modeling of the interactions between multiple states within a region.  






\section{A Diversion Scenario: Highly Enriched Uranium}
\label{s_results}


% TODO: Actual #s: qty HEU, qty LEU, Power units, gaussian params etc.

% TODO: DIVERSION SCENARIO LAYOUT
\begin{figure}%[htbp!]
\begin{center}
\includegraphics[natwidth=162bp,natheight=227bp, scale=0.6]{./figs/**.png}
\end{center}
\caption{TODO:CAPTION}
\label{fig:scenario_layout}
\end{figure}


As a part of the \gls{CVT}\footnote{http://cvt.engin.umich.edu/}, \Cyclus is being used to generate multi-modal datasets with signatures of diversion with the goal of improving anomaly detection techniques. In the simplest implementation, \gsls{HEU} is clandestinely produced and then diverted from an enrichment facility.  Figure \ref{fig:scenario_layout} illustrates a toy model of this portion of the fuel cycle. A facility such as a mine supplies natural uranium (0.7\% $^{235}$U) to an enrichment facility. The enrichment facility in turn receives requests for \gls{LEU} of various enrichments from several declared light-water reactors (ignoring the fuel fabrication facility for simplicity).  The enrichment facility also receives requests for 90\% enriched \gls{HEU} from an undeclared actor seeking to build a nuclear weapon. Material production for each facility is calculated once each month for a total of 100 months. At each timestep, the enrichment facility fulfills an order for one \gls{LEU} enrichment level, and sometimes produces small quantities of \gls{HEU} request. 

% TODO: Double-plot: Power consumption and LEU production (with HEU production echoed on plot)
% ~/git/data_analysis/data/UM_data/multi_modal_v1.3/single_runs/INMM_051316

\begin{figure}%[htbp!]
\begin{center}
\includegraphics[natwidth=162bp,natheight=227bp, scale=0.6]{./figs/**.png}
\end{center}
\caption{TODO:CAPTION}
\label{fig:time_series}
\end{figure}

Figure \ref{fig:time_series} shows the time series data for declared production of \gls{LEU} (top) and the total SWU consumed by the enrichment plant (bottom) available to the inspector (SWU can be used as a rough proxy for power consumption).  Months where \gls{HEU} is produced are denoted with green on the \gls{LEU} plot.  The \gls{HEU} signature is hidden in the material flow data because there is a gaussian variation (TODO: PARAMS) in the tails assay.  Therefore it is not possible to detect diversion from the individual time-series data alone. However, the two signals can be combined to highlight the correlated signature of diverion. Figure \ref{fig:power_qty} shows time-series data for the ratio of power-consumption to declared \gls{LEU} production.  When the variation due to tails assay (``noise'') is sufficiently small, the deviations can be seen by eye (TODO: FOR EXAMPLE, AT T= ???).

% TODO: Ratio of LEU/POWER showing clear signature of diversion
\begin{figure}%[htbp!]
\begin{center}
\includegraphics[natwidth=162bp,natheight=227bp, scale=0.6]{./figs/**.png}
\end{center}
\caption{TODO:CAPTION}
\label{fig:time_series}
\end{figure}

While this example is clearly a toy model, it illustrates the power of combining multiple signals from a scenario to improve detection capabilities. In practice, as the noise increases or the \gls{HEU} quantity reduces in amplitude, this signature quickly becomes difficult to detect by eye. An important application of \Cyclus is to produce more complex synthetic datasets which are then provided to groups that specialize in developing advanced anomaly detection techniques. An ongoing collaboration with researchers at the Michigan Institute for Data Science\footnote{http://midas.umich.edu/} seeks to apply innovative anomaly detection techniques to these simulations to investigate detection limits for scenarios with sparse data sets or low signal-to-noise ratios.


\section{Discussion}
\label{s_dis}

\Cyclus is able to model signatures of diversion from a diverse set of facilities in the nuclear fuel cycle and with a variety of data modalities. Table \ref{tab:modalities} lists a variety of signal modalities and their applications in the fuel cycle (TODO: synonym for schema?).  One modality with diverse set of potential applications is satellite imagery.  We are now developing the software infrastructure to create synthetic satellite images that may contain signals of diversion. Satellite imagery has a variety of applications: tracking personnel or truck movement patterns, thermal or visible signatures of effluent or heat, or major facility changes such as new or removed buildings.

%% TODO: MAKE TABLE WITH MODALITIES AND THEIR APPLICATIONS
% Simulation parameters in RS_3sink.xml at
%/Users/mbmcgarry/git/data_analysis/data/v1.2/random_sink/
\begin{table}
\centering
\begin{tabular}{|c|c|c|}
\hline
\textbf{General}    & Duration (months)       & 100  \\
\textbf{Simulation} & Natural U (\% $^{235}U$) & 0.7  \\
\textbf{Parameters} & LEU (\% $^{235}U$)       & 4.0  \\
                    & HEU (\% $^{235}U$)       & 90.0 \\
\hline
\textbf{Enrichment} & SWU Capacity (kg-SWU/month) & 180  \\
\textbf{Facility}   & Tails Assay (\% $^{235}U$)   & 0.3  \\
\hline
\textbf{LEU Demand} & Mean Qty (kg)       & 33.0  \\
                    & $\sigma$ (kg)       & 0.5  \\
\hline
\textbf{HEU Demand} & Qty (kg)            & 0.03  \\
                    & Avg Rate of Occurrence & 1/5 \\ 
\hline
\end{tabular}
\caption{Simulation parameters for \gls{HEU} diversion scenario.}
\label{tab:sim_params}
\end{table}

These diverse datasets can be combined to highlight signatures of diversion that are small enough to be hidden in the noise of individual signals.  We have illustrated this technique by combining time-series data for power consumption and declared \gls{LEU} production for a simple scenario of \gls{HEU} production in an enrichment facility.  More realistic scenarios require advanced anomaly detection techniques such as those being developed at \gls{UM}. A collaboration with \gls{UM} and \gls{Sandia} will investigate ways to optimize subsets of diverse signal modalities to ensure reliable detection while minimizing resource usage.

The \Cyclus fuel cycle simulator is being used as a framework for combining techniques and knowledge from a variety of disciplines to support a cohesive approach to treaty verification.  Moving forward, \Cyclus will be used to study more complex and realistic diversion scenarios.  Additionally, \Cyclus has the capability to produce synthetic signals of inherent physical processes such as neutron spectra of various materials.  In this way, \Cyclus simulations can provide theoretical signals to researchers developing experimental detectors in order to test sensitivity and detector response.  \Cyclus is also being used to explore behavioral mode ***


Probabilistic models for behavior based on the actor's risk-perception will be explored.  Ongoing collaborations as part of the \gls{CVT} are examining the mechanisms and limits of expanding anomaly detection algorithms with other types of data, such as social media chatter.  Due to the inherently interdisciplinary nature of this work, new external collaborations are sought with experts in behavioral modeling. Innovative ideas on detection modalities and diversion detection techniques are also welcomed.



\textit{This work was funded in-part by the Consortium for Verification Technology under Department of Energy National Nuclear Security Administration award number DE-NA0002534”}


%%%%%%%%%%%%%%%%%%%%%%%%%%%%%%%%%%%%%%%%%%%%%%%%%%%%%%%%%%%%%%%%%%%%%%%%%%%%%%%%
\bibliographystyle{ans}
\bibliography{/Users/mbmcgarry/git/papers/Master}
\end{document}
