\section{The \Cyclus Fuel Cycle Simulator}
\label{s_methods}

The \gls{CNERG}\footnote{http://cnerg.github.io/} group at the University of Wisconsin has developed the \Cyclus\footnote{http://fuelcycle.org/} nuclear fuel cycle simulator to model all aspects of the nuclear fuel cycle in a flexible way \cite{cyclus_v1_0,cyclus_v1_2,cyclus_v1_3}. \Cyclus produces a database file containing information on the flow of nuclear material through the fuel cycle at each timestep.  The database provides information on facility inventories, material composition, transactions between facilities, and facility build and decommissioning histories, among others.

\Cyclus is designed using an agent-based framework, meaning that each actor  in a fuel cycle (such as a mine, a nuclear reactor, or even a governing body) is modeled as an independent agent \cite{jennings_agent-based_2000, taylor2014agent}.  Each agent in the simulation is self-contained and may include physics, economics, or behavioral components \cite{huff_open_2011,gidden_agent-based_2013,gidden_agent-based_2015}.  The agents interact with one another through the \gls{DRE}, which facilitates the trading of resources and commodities \cite{gidden_agent-based_2014}.  At each timestep, agents can choose to request resources.  Resources are defined using both a quantity (e.g. 1 metric ton), and a quality, such as having a composition of 99.7\% $^{238}U$ and 0.3\% $^{235}U$.  The \gls{DRE} then solicits bids from any facilities that are interested in offering those resources. Resources can be offered as bids even if they do not exactly match the requested material. For example, a reactor might request a commodity called ``fuel'', which it has defined as being \gls{UOX}.  It may receive two bids for ``fuel'' that are specified as \gls{UOX} and \gls{MOX}, having two distinct isotopic compositions. After the bids are received, the requestor is able to state a preference for one bid over another. Finally, once the preferences have been applied, the \gls{DRE} calculates all potential trades across all agents, then executes an optimization algorithm to find the solution that most closely fulfills a maximum of requests. It is possible that as a result of the preference adjustment, no trades are executed for some facilities in a given time step.  Once this solution is found, material is transfered across the facilities and the timestep is concluded.

Although \Cyclus was designed to assess the transition from once-through fuel cycles to alternative next-generation scenarios including technologies such as spent fuel recycling, it is also an excellent tool for examining proliferation issues.  Other fuel cycle simulators designed to study energy issues are typically too inflexible to be used for other purposes. For example, they may have extremely accurate physics models of reactor burnup, but no ability to vary or modify other facilities in the fuel cycle.  \Cyclus has three key features that make it well-suited to non-proliferation studies: it is \textit{agent-based}, it incorporates \textit{social and behavioral interaction models}, and it tracks \textit{discrete materials}.

\subsection{Agent-Based}
Agent-based design is important in studying non-proliferation because it allows the comparison of different scenarios by changing just one aspect of the simulation at a time.  \Cyclus features a modular design in which individual facility models can be swapped in otherwise identical fuel cycles for comparison. For example, a user could compare enrichment technologies by creating two different enrichment modules, one using gaseous diffusion and the other using centrifuge technology. Then two simulations can be run where the entire fuel cycle is identical except for the two different enrichment designs, and the results can therefore be directly compared. The other benefit of the agent-based model is that simulations can be made as detailed facility models, generalized systems-level models, or a hybrid of both. This facilitites the studying of how changes in one part of the system will impact other areas.  In proliferation studies, this means that the effects on all other facilities from diversion at a single site can be modeled simultaneously.

\subsection{Behavioral Modeling}
The preference adjustment phase of the \gls{DRE} allows for the introduction of interaction behaviors.  Behavioral modeling is critical to studying non-proliferation because it can be used to approximate real-world motivations and interactions that are not captured by economics or other conventional decision-making models.  Each agent can prioritize bids for resources in any way it chooses. A specific agent might have preferences based on material composition, physical proximity between facilities, or allowed and disallowed trading partners, which are implemented in a region-institution-facility hierarchy that enables economic modeling \cite{oliver_geniusv2:_2009}.  Individual facilities can be managed by insistutions, such as multinational corporations, utilities, government agencies, etc.  Facilities and institutions can then be ascribed to distinct regions, which may represent geographical regions, political alliances or economic trading partners. This feature allows \Cyclus to model tariffs, sanctions and other types of economic agreements.    Agents may also make decisions about interacting at each timestep based on their own internal logic, for example, an illicit facility may choose not to trade at every timestep in an attempt to avoid detection. 

\subsection{Discrete Materials}
\Cyclus tracks discrete material flow through the simulation, meaning that once a material enters the simulation, its location and quality is tracked at all remaining timepoints \cite{huff_integrated:_2013}. Discrete material modeling is useful in studying non-proliferation because material found anywhere in the fuel cycle can be tracked back to its source, providing attribution for illicit behavior. \Cyclus includes nuclear data from PyNE,\footnote{http://pyne.io/} a computational nuclear science tool that enables calculation of decay and transmutation \cite{Scopatz2012b}. As a result, \Cyclus can  model decay of materials and track all decay products from a parent isotope, facilitating studies of heat loading, radiation exposure, and other derived fuel cycle metrics \cite{scopatz_cymetric_2015}. This also opens to the door to using \Cyclus to model historical events, with applications in nuclear forensics and archeology.
