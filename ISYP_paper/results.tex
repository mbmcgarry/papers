\section{Technical Results}
\label{s_results}

\subsection{\gls{HEU} Diversion}
\label{s_heu}
\Cyclus has been used to incorporate social-behavioral modeling into simulations
of \gls{HEU} diversion from a declared fuel cycle.  In the simplest implementation, material is diverted at the enrichment facility.  Figure \ref{fig:heu_layout} illustrates a toy model of this portion of the fuel cycle. A facility such as a mine supplies natural \gls{U} (0.7\% $^{235}$U) to an enrichment facility.  The enrichment facility in turn receives requests for 4\% enriched \gls{LEU} from a declared light-water reactor (ignoring the fuel fabrication facility for simplicity).  The enrichment facility also receives requests for 90\% enriched \gls{HEU} from an undeclared actor seeking to build a nuclear weapon. Material flow, or throughput, out of each facility is is calculated once each month for a total of 100 months.  This framework allows us to pose the following question: If an inspector only has access to the inventory records for LEU arriving at the delared reactor, can diversion of material be identified?

\begin{figure}%[htbp!]
\begin{center}
\includegraphics[natwidth=162bp,natheight=227bp, scale=0.7]{./figs/heu_cyclist_layout.png}
\end{center}
\caption{Agent layout and material flow for a toy model of \gls{HEU} diversion from a declared fuel cycle at the enrichment facility.}
\label{fig:heu_layout}
\end{figure}

To make the scenario more realistic, three conditions are applied to the simulation:
\begin{itemize}
\item{The enrichment facility is nominally operating near its \gls{SWU} limit}
\item{The demand for \gls{LEU} material has a time-varying amplitdue}
\item{The enrichment facility will always prioritize fulfilling requests for \gls{HEU}}
\end{itemize}
The \gls{SWU} limit, a metric that incorporates power-consumption and maximum processing throughput, constrains the simulation so that if the enrichment facility chooses to produce \gls{HEU} then its \gls{LEU} output will necessarily decrease.  As a result, if the \gls{LEU} demand were constant then there would be a clear signature of diversion when \gls{HEU} was produced and the simulation would be trivial.  Moreover, the time-variation in \gls{LEU} demand is more representative of real-life, where a single enrichment facility provides fuel to many reactors, which may operate on different reloading schedules. It is also assumed that there will be variations in demand due to unanticipated reactor shutdowns, delays in receiving raw material, maintenance and repairs, etc.  While these events are somewhat mitigated in real operations by the use of long-term contracts and material reserves, a small variation in nominal \gls{LEU} demand is a reasonable assumption.

% Simulation parameters in RS_3sink.xml at
%/Users/mbmcgarry/git/data_analysis/data/v1.2/random_sink/
\begin{table}
\centering
\begin{tabular}{|c|c|c|}
\hline
\textbf{Gen. Sim.} & Dur. (months)      & 100  \\
                    & Nat. U \% $^{235}U$ & 0.7  \\
                    & LEU \% $^{235}U$    & 4.0  \\
                    & HEU \% $^{235}U$    & 90.0 \\
\hline
\textbf{Enrich. Fac.} & SWU Capacity       & 180  \\
                    & tails assay (\%)   & 0.3  \\
\hline
\textbf{LEU Demand} & Mean Qty (kg)       & 33.0  \\
                    & $\sigma$ (kg)       & 0.5  \\
\hline
\textbf{HEU Demand} & Qty (kg)            & 0.03  \\
                    & Avg Occur. (months) & 1/5 \\ 
\hline
\end{tabular}
\caption{Simulation parameters for \gls{HEU} diversion scenario.}
\label{tab:sim_params}
\end{table}

Two different behavioral models have been used for the behavior of the illlicit actor who is requesting \gls{HEU}. In one scenario, the illicit actor requests a small quantity of \gls{HEU} at a regular interval.  In the other scenario, the actor also asks for the same quantity of material in each request, but the requests do not have a constant frequency. Rather they are modeled as occuring `randomly' with a gaussian distribution that defines their average rate of occurrence.
Table \ref{tab:sim_params} lists the parameters used for the simulation presented here, where \gls{HEU} is diverted using the `random' model with an average occurrence rate of once per 5 months.  Figure \ref{fig:heu_demand} shows the \gls{HEU} demand from the illicit actor as a function of time.  

% Figures made with ~/git/data_analysis/fourier_analysis.ipynb, and saved in
% /Users/mbmcgarry/git/data_analysis/data/v1.2/random_sink/png
\begin{figure}
\begin{center}
\includegraphics[natwidth=162bp,natheight=227bp, scale=0.7]{./figs/HEU_R5.png}
\end{center}
\caption{An illicit actor request \gls{HEU} from the enrichment facility at random timesteps with an average occurrence rate of 1 out of every 5 months}
\label{fig:heu_demand}
\end{figure}

\begin{figure}
\begin{center}
\includegraphics[natwidth=162bp,natheight=227bp, scale=0.7]{./figs/nat_delta_R5.png}
\end{center}
\caption{Actual \gls{LEU}The natural variation of \gls{LEU} demand in the absence of diversion is shown in orange, while the \gls{LEU} actually produced in the diversion scenario is shown in blue. Amplitude of diverted material was chosen to displace the natural signal by 1-$\sigma$ the natural variation.}
\label{fig:leu_produced}
\end{figure}

Figure \ref{fig:leu_produced} illustrates the impact of this illicit material diversion on overall \gls{LEU} production.  The orange trace represents the amount of \gls{LEU} material that would have been produced by the enrichment facility at each timestep if there had been no material diversion for \gls{HEU} production.  The blue trace represents the amount of \gls{LEU} that was actually produced and shipped to the declared reactor during the diversion scenario.

\begin{figure}
\begin{center}
\includegraphics[natwidth=162bp,natheight=227bp, scale=0.7]{./figs/netLEU_hist_R5_new.png}
\end{center}
\caption{The natural variation of \gls{LEU} demand in the absence of diversion is shown in orange, while the \gls{LEU} actually produced in the diversion scenario is shown in blue. Amplitude of diverted material was chosen to displace the natural signal by 1-$\sigma$ the natural variation.}
\label{fig:leu_histogram}
\end{figure}

If the inspector has access to only the blue time-series data, is it possible for them to identify whether or not material is being diverted from the system?  A naive approach is to look at the distribution of the timeseries.  Figure \ref{fig:leu_histogram} is a histogram of the declared \gls{LEU} signal.  The blue trace is a gaussian fit to the declared data, while the orange trace is a gaussian fit to the hypothetical variation in the \gls{LEU} signal if there were no diversion.  In this example, an inspector would need a sufficiently well-sampled dataset for the facility during a time when there was no diversion as a baseline to determine the expected type of distribution as well as a baseline for its qantitative features in order to draw conclusions about whether or not material diversion is occuring. An ongoing collaboration with researchers at the Michigan Insitute for Data Science seeks to apply innovative anomaly detection techniques and multi-modal data integration to these simulations to successfully identify material diversion with sparse data sets or low signal-to-noise ratios \cite{HERO_research}.






  